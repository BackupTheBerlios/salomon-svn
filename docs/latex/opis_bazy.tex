\section{Opis bazy danych}sz

Do przechowywania danych wykorzystywanych przez system
wykorzystana zosta�a baza danych. Dane zosta�y zorganizowane w
nast�puj�cych tabelach:

\paragraph{Projects}

Tabela zawiera nag��wki projekt�w, do kt�rych odnosz� si� rekordy
z tabeli Tasks.

\begin{itemize}
    \item \emph{project\_id} - identyfikator projektu
    \item \emph{name} - nazwa
    \item \emph{info} - dodatkowy opis
\end{itemize}

\paragraph{Plugins}
Tabela zawiera informacje o pluginach, kt�re mog� by� wykorzystane
przez system. Przechowuje dane o ich nazwach i lokalizacjach, sk�d
mog� by� pobrane.
\begin{itemize}
    \item \emph{plugin\_id} - identyfikator pluginu
    \item \emph{name} - nazwa
    \item \emph{info} - dodatkowy opis
    \item \emph{location} - lokalizacja pluginu.
\end{itemize}

\paragraph{Tasks}
 Tabela zawiera zapis wykonania poszczeg�lnych task�w . Dla ka�dego
zadania przechowywane s� dane o pluginie, kt�ry zosta�
wykorzystany do wykonania zadania, projekcie, w ramach kt�rego
zadanie zosta�o zapisane, ustawienia, z jakimi zosta�o wykonane,
rezultat zadania oraz status, w jakim pozostaje po wykonaniu.

\begin{itemize}
    \item \emph{task\_id} - identyfikator taska
    \item \emph{project\_id} - identyfikator projektu, do kt�rego nale�y task
    \item \emph{plugin\_id} - identyfikator pluginu, kt�ry wykonywany jest w ramach tego taska
    \item \emph{name} - nazwa
    \item \emph{info} - dodatkowy opis
    \item \emph{plugin\_settings} - ustawienia pocz�tkowe pluginu
    \item \emph{plugin\_result} - rezultat dzia�ania pluginu
    \item \emph{status} - status wykonania taska
\end{itemize}

\paragraph{Datasets}
Zawiera nag��wki zbior�w zada�.
\begin{itemize}
    \item \emph{dataset\_id} - identyfikator zbioru danych
    \item \emph{dataset\_name} - nazwa
    \item \emph{info} - dodatkowy opis
\end{itemize}

\paragraph{Dataset\_items}
Zawiera definicje zbior�w zada�. Zbi�r danych definiowany jest
przez nazw� tabeli oraz warunki, jaki ograniczaj� rekordy w tej
tabeli.
\begin{itemize}
    \item \emph{dataset\_item\_id} - identyfikator elementu zbioru danych
    \item \emph{dataset\_id} - identyfikator zbioru danych do kt�rego nale�y
element
    \item \emph{table\_name} - nazwa tabeli
    \item \emph{condition} - warunek ograniczaj�cy zakres danych
\end{itemize}
    Relacje mi�dzy nimi zobrazowane s� na
poni�szym rysunku:
