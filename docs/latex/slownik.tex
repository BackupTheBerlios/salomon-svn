\section{S�ownik}

\paragraph{Kontroler}
Kontroler to jedna instancja Salomona pracuj�ca w okre�lonym trybie: lokalnym, jako serwer lub klient. 

\paragraph{Manager} Klasy maj�ce w nazwie s�owo \emph{Manager} pe�ni� szczeg�ln� funkcj� - najcz�ciej odpowiadaj� za zarz�dzanie innymi klasami, z regu�y tymi, kt�re stanowi� reszt� nazwy (np. \emph{PluginManger} zarz�dza pluginami, \emph{ProjectManger} - projektami)

\paragraph{Projekt}
W ramach projekt�w zapisywana jest konfiguracja systemu przed wykonaniem kolejki zada�. Zapisywana jest lista zada� do wykonania, ich ustawienia oraz pluginy, kt�re maj� by� u�yte do ich wykonania.

\paragraph{Plugin}
Jest to zewn�trzny program odpowiedzialny za wykonanie konkretnego zadania. Nie jest integraln� cz�ci� systemu, w razie konieczno�ci jest pobierany z okre�lonej lokalizacji i przekazywane jest mu konkretne zadanie do wykonania.