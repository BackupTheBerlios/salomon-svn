\section{Tworzenie W�asnego Pluginu}
	System, dzi�ki swej elastycznej architekturze, pozwala u�ytkownikowi na rozszerzanie jego mo�liwo�ci poprzez definiowanie przez niego w�asnych plugin�w. Ca�a funkcjonalno�� zwi�zana z przetwarzaniem plugin�w, konfigurowaniem ich, prezentacj� wynik�w ich dzia�ania oraz komunikacj� przez sie� ukryta jest przed tw�rc� pluginu i nie musi by� brana pod uwag� przy jego projektowaniu i implementacji. Jedyne co musi on zrobi� to zaimplementowa� opisane wy�ej interfejsy.
By system m�g� skorzysta� z nowych plugin�w nale�y doda� do tabeli plugins odpowiedni wpis, umo�liwiaj�cy systemowi zlokalizowanie i �ci�gni�cie pluginu. 
Obecna wersja systemu obs�uguje tylko pluginy w postaci archiw�w jar.
