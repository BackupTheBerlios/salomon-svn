\section{Drzewa decyzyjne}
% http://www.dbmsmag.com/9807m05.html
% When a businessperson needs to make a decision based on several factors, a
% decision tree can help identify which factors to consider and how each factor
% has historically been associated with different outcomes of the decision. For
% example, in our credit risk case study (See the sidebar Predicting Credit Risk),
% we have data for each applicant�s debt, income, and marital status. A decision
% tree creates a model as either a graphical tree or a set of text rules that can
% predict (classify) each applicant as a good or bad credit risk.

Kiedy biznesmen potrzebuje podj�c dezycje bazuj�c na kilku wsp�cznynnikach,
drzewa decyzyjne mog� pomo� w wybraniu odpowiednich wsp�czynnik�w oraz 
opowiedzie� jak dany wsp�czynnik w przesz�o�ci wp�ywa� na decyzje.
Przyk�adowo, podczas obliczania ryzyka kredytowego, posiadamy dla ka�dego
potencialnego kredytobiorcy dane o jego kredytych, przychodach oraz jego
statusie materialnym. Drzewa decyzyjne tworz� zar�wno drzewo w postaci
graficznej, jak zbi�r regu� tekstowych, dzi�ki kt�rym mo�emy okre�li�,
czy dana aplikacja jest obarczona du�ym lub ma�ym ryzykiem kredytowym.

% A decision tree is a model that is both predictive and descriptive. It is called
% a decision tree because the resulting model is presented in the form of a tree
% structure. (See Figure 1.) The visual presentation makes the decision tree model
% very easy to understand and assimilate. As a result, the decision tree has
% become a very popular data mining technique. Decision trees are most commonly
% used for classification (predicting what group a case belongs to), but can also
% be used for regression (predicting a specific value).



\subsection{Om�wienie algorytm�w}
O
\subsubsection{ID3}
I
\subsubsection{C4.5}
C