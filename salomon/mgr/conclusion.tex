Celem pracy by�o, om�wienie, zaproponowanie architektury, oraz dostarczenie przyk�adowej implementacji indukcyjnej bazy danych jako platformy uczenia maszynowego. Dzi�ki przeprowadzony eksperymentom zosta�o dowiedzione, �e platforma Salomon stanowi wyj�tkowo proste narz�dzie zar�wno od strony procesu tworzenia wtyczek jaki zarz�dzania i uruchamia zada�.

W ramach pracy zaprojektowano oraz zrealizowano system bazuj�cy na zaproponowanej architekturze. Z postulat�w napisany w rozdziale drugim uda�o si� zrealizowa�


Bardzo istotnym sk�adnikiem platformy, kt�ry nie zosta� zrealizowany w ramach tej pracy jest obs�uga �rodowiska rozproszonego. Wsparcie dla rozproszenie i zr�wnoleglenia oblicze� jest najwa�niejszym zadaniem w planowanych dalszych pracach na rozwojem platformy. Dodatkowo w ramach dalszych system zostanie rozszerzony poprzez dodanie wsparcia dla innych metod zapisu wiedzy takich jak regu�y, czy klastry.





%a) Celem pracy by�o, zrobiono, zrealizowano, zaproponowano
%b) Najwa�niejsze osi�gni�cia. Nawi�za� do rozdzia�u drugiego
%c) Plany rozwoju

%Celem pracy by�o, om�wienie, zaproponowanie architektury,
%oraz dostarczenie przyk�adowej implementacji indukcyjnej bazy
%danych jako platformy uczenia maszynowego. 
%
%Dalsze prace nad rozwojem system Salomon powinny koncentrowa� si�
%na dodanie wsparcia dla pracy w �rodowisku rozproszonym.




%W pracy zaprezentowano system rozproszonych obliczen� ewolucyjnych bazuja�cy
%na stadnym modelu ewolucji. Koncepcj�e system oparto na analogiach do proces�w zachodza
%�cych w populacjach organizm�w z�ywych.
