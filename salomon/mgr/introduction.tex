%-------- Tre�� wst�pu --------%

%a) Og�lnie (Mamy du�o metod ....., wiele system�w ....)
%b) Cel pracy (koncepcja, realizacja)
%c) Plan pracy. Co w kt�rym rozdziale. Po jednym zdaniu na rozdzia�


Pierwszy rozdzia� pracy stanowi wprowadzenie do problematyki uczenia maszynowego -- opisane s� w nim przyk�adowe metody reprezentacji wiedzy, algorytmy s�u��ce do wydobywania wiedzy oraz przyk�adowe oprogramowanie do uczenia maszynowego.
Drugi rozdzia� koncentruje si� na architekturze systemu -- przedstawiono w nim koncepcj� systemu, jego struktur� oraz funkcjonalno��. Kolejny rozdzia� prezentuje wybrane aspekty implementacji systemu -- jego podzia� na platform�, kontrolery i wtyczki wraz z opisem g��wnych interfejs�w, za pomoc� kt�rych g��wne cz�ci systemu wsp�pracuj� ze sob�. Rozdzia� czwarty stanowi opis graficznego interfejsu u�ytkownika, za pomoc� kt�rego obs�ugiwany jest system. W ostatnim rozdziale zaprezentowane zosta�o przyk�adowe wykorzystanie systemu do rozwi�zania zagadnienia uczenia maszynowego.
