\documentclass{mgragh} % opcje: robocza,man
\usepackage[cp1250]{inputenc}  % opcja latin2 dla Linuxa lub cp1250 dla Windows
\usepackage[polish]{babel}
\usepackage[OT4]{fontenc}
\usepackage{polski}
%%
%%
\makeindex 
% \includeonly{Mgr_pom,Mgr_wst,Mgr_roz1,Mgr_roz2,Mgr_lit}

\bibliographystyle{ddabbrv}
%\nocite{*}

\begin{document}
%%
%%
%% ======== METRYCZKA PRACY ========
\title{O budyniu tyczkowym}
\author{Nikodem }
\promotor{prof. dr hab. Kongruencja Kowariantna}
\nralbumu{137}
\uczelniaNazwa{Akademia G�rniczo-Hutnicza}
\uczelniaImienia{im. Stanis�awa Staszica}
\wydzial{Wydzia� Monterstwa Sprytnego}
\kierunek{Atryku�y biurowe}
\specjalnosc{Skr�canie d�ugopis�w}
\rok{2006}

\maketitle
%%
\slowakluczowe{maciejka, nafta, serek}

%\keywords{a,b,c,d,e,f,g,h} 
%%
%%
%% ======== NASZE MAKRA ========
%%
%% ************ AKADEMIA G/ORNICZO-HUTNICZA W KRAKOWIE *************
%% ***************** Wydzial Matematyki Stosowanej *****************
%% ****************** PRACA MAGISTERSKA w LaTeX-u ******************
%%    autor: ------
%%    Copyright (C) 2002 by ------
%% ********************* Plik z definicjami ************************
%%

%----------------- nasze definicje -----------------% 
\newtheorem{stw}{\indent Stwierdzenie}[chapter]
%------------------------------%
\newcommand{\id}[1]{\index{#1}}  
\newcommand{\wi}[1]{#1\index{#1}}  
\newcommand{\wwi}[1]{\emph{#1}\index{#1}}  
\newcommand{\mwi}[1]{\textbf{#1}\index{#1}}
\newcommand{\ii}[1]{\textit{#1}}
%\newcommand{}{}

%---------------------------------------------------%



%%
%% ======== SPIS TRE�CI ========
%%
\tableofcontents
%%
%% ======== STRESZCZENIE PRACY (POLSKIE) ========
\begin{streszczenie}
%%
\documentclass[a4paper,12pt]{article}
\usepackage[polish]{babel}
\usepackage[cp1250]{inputenc}
\usepackage[T1]{fontenc}
\usepackage[dvips]{graphicx}
\usepackage{fancyvrb}

\usepackage[urlcolor=blue, colorlinks=true, bookmarks=true, bookmarksnumbered=true]{hyperref}
\usepackage{indentfirst}
\usepackage{times}
\usepackage{epsfig}
\usepackage{color}

\usepackage{fancyhdr} 

\pagestyle{fancy} 

\rfoot{SALOMON}


%ustawienie wielkosci akapitu
%\setlength{\parindent}{0mm}
%%odstep miedzy akapitami
%\setlength{\parskip}{2mm}

%ustawienia rozmiaru tekstu
%    \textwidth 16cm
%    \textheight 24cm 
%    \topmargin -15mm
%    \evensidemargin-3mm
%    \oddsidemargin -3mm   

\title{\emph{Salomon} extended abstract}

\author{Nikodem Jura \emph{(nico@icslab.agh.edu.pl)}
        \\
        Krzysztof Rajda \emph{(krzycho@student.uci.agh.edu.pl)}
				}                       

\begin{document}

\maketitle
%wylaczenie numerowania na I stronie
\thispagestyle{empty}

\begin{center}
\textcolor{blue}{\href{http://salomon.iisg.agh.edu.pl}{http://salomon.iisg.agh.edu.pl}}
\end{center}

\begin{center}
Prowadz�cy: dr in�. Marek Kisiel-Dorohinicki
\end{center}
\emph{Knowledge Mining} to nowo rozwijana metodologia, kt�ra pozwala wydobywa� u�yteczn� wiedz� zar�wno z danych jak i z wiedzy og�lnej. W metodologii tej bardzo du�y nacisk k�adziony jest na prezentacj� wynik�w w spos�b
przyjazny dla cz�owieka, np. za pomoc� j�zyka zbli�onego do j�zyka naturalnego, oraz za pomoc� r�norodnych graficznych reprezentacji. Pozwala one na generowanie wiedzy zar�wno z wielkich system�w bazodanowych gdzie dane mog� by� niekompletnych a nawet cz�ciowo b��dne.

Jednym z naukowc�w zajmuj�cych si� dziedzin� \emph{Knowledge Mining} jest \emph{prof. Ryszard S. Michalski}. Wraz z innymi naukowcami z \emph{George Mason University (USA)} stworzy� on system \emph{Vinlen}, kt�rego zadaniem jest realizowanie zada� metodologii \emph{Knowledge Mining} wykorzystuj�c indukcyjne bazy danych. Celem tego projektu jest integracja wielu istniej�cych metod wydobywania wiedzy takich jak np. \emph{AQ} (algorytm generacji regu� atrybutowych), czy metod konceptualnego klastrowania. W systemie zosta�y r�wnie� zawarte mechanizmy zarz�dzania danymi, operatory wizualizacji wiedzy (ITG) oraz tworzenie graf�w powi�za�. W celu �atwego u�ytkowania systemu zosta� on wyposa�ony w przyjazny graficzny interfejs u�ytkownika.

Z zwi�zku z pewnymi ograniczeniami architektury \emph{VINLEN} powsta� projekt maj�cy na celu stworzenie dla niej alternatywy pozbawionej tych ogranicze�.

\begin{itemize}
	\item \textbf{Ca�a funkcjonalno�� umieszczona we wtyczkach.}
	Silnik programu powinnen by� tak lekki jak tylko to mo�liwe. Jego zdaniem jest dostarczenie przyjaznego �rodowiska do realizacji funkcjonalno�ci zawartej we wtyczkach.
	
	\item \textbf{Otwarta architektura}
	Wprowadzenie warstwy izoluj�cej wtyczyki od bazy danych. Otwarta architektura pozwala na wygodne rozszerzanie tej warstwy.
	
	\item \textbf{Niezale�no�� od platformy}
	System powinnen by� niezale�ny od platformy. Powinna by� mo�liwo�� pracy w rozproszonychh hetegeronicznym �rodowisku.
	\item \textbf{Mo�liwo�� definowania przep�ywu sterowania}
	\item \textbf{�atwo�� adaptowania funkcjonalno�ci zawartej w systemie \emph{VINLEN}}
	
\end{itemize}

Ekstrakcja wiedzy to proces iteracyjny podzielony na etapy, kt�re mog� tworzy� cykle.

W ka�dym tym etapie wiedza jest poddawana obr�bce. Na ka�dym takim etapie proces odkrywania mo�e by� ukierunkowany zgodnie
z wymaganiami u�ytkownika. Ka�dy etap tworzy odr�bn� ca�o��. Cz�� etap�w jest roz��czna, zatem mog� by� wykonywane wsp�bie�nie. Architektura Salomona pozwala na rozproszenie, a co za tym idzie -- zr�wnoleglenie, wykonywania takich roz��cznych etap�w oraz na synchronizacj� ich wynik�w.

W Salomonie reprezentacj� etapu jest \emph{zadanie} (Task). Ka�de zadanie mo�e posiada� kilka nast�pnik�w i kilka
poprzednik�w. Ujmuj�c rzecz pro�ciej, zadania mog� by� zorganizowane w postaci grafu. 

Zadanie zdefiniowane jest jako tr�jka:
\begin{itemize}
	\item algorytm(dostarczany w postaci wtyczki, czyli \emph{pluginu}))
	\item dane steruj�ce algorytmem
	\item dane wej�ciowe, z kt�rych algorytm b�dzie wydobywa� wiedz�
\end{itemize}

Na wej�ciu lub wyj�ciu ka�dego zadania mog� pojawi� si� dane (w przypadku algorytm�w, kt�re
dokonuj� selekcji/segregacji danych), wiedza lub obie te rzeczy naraz.

Salomon sk�ada si� z dw�ch zasadniczych cz�ci (Rys \ref{fig:Achitecture}). Pierwsz� z nich
stanowi platforma. Tworz� j�:
\begin{itemize}
	\item silnik, kt�rego zadaniem jest zarz�dzanie i~uruchamianie
zada�
	\item magazyn s�u��cy do przechowywania wiedzy
\end{itemize}

Druga cz�� to zbi�r wtyczek, dostarczaj�cych logik� potrzebn� do skonfigurowania, wykonania i~wy�wietlenia rezultat�w
zadania. System jest otwarty - oznacza to, �e jego funkcjonalno�� mo�e by� w �atwy spos�b rozszerzana poprzez dostarczenie nowych wtyczek.


%spis tresci
%\tableofcontents
%\newpage
%VINLEN inductive database system is brie
%y
%reviewed and illustrated by selected results. The goal of research on VINLEN is to
%develop a methodology for deeply integrating a wide range of knowledge generation
%operators with a relational database and a knowledge base. The current system
%has already integrated an AQ learning system for generating attributional rules in
%two modes: theory formation, in which generated rules are consistent and complete
%with regard to data, and pattern discovery, in which generated rules represent strong
%patterns, not necessarily consistent or complete. It also has integrated a conceptual
%clustering module for splitting data into conceptual classes, and providing descrip-
%tions of those classes. Preliminary data management and knowledge visualization
%operators, such as the intelligent target data generator (ITG) and concept asso-
%ciation graph display, have also been integrated. To facilitate an easy interaction
%with the system, a user-oriented visual interface has been implemented. An exam-
%ple of results from applying VINLEN to a medical problem domain is presented to
%illustrate VINLEN knowledge discovery and representation capabilities.



%\newpage
%Dziedzina baz danych znajduje si� obecnie w fazie niespotykanego rozwoju -- staj� si� one nie tylko wszechobecne, ale i~globalnie po��czone. W zwi�zku z tym pojawi�o si� wiele nowych kierunk�w ich rozwoju, jednym z nich jest \emph{Data Mining}.
%
%\emph{Data Mining} to dziedzina nauki, zajmuj�ca si� przetwarzaniem danych zmagazynowanych w bazach danych w celu wydobycia z nich wiedzy. W przeciwie�stwie do konwencjonalnych baz danych, wiedza ta umo�liwia znalezienie odpowiedzi nie tylko na pytania, na kt�re odpowiedzi znajduj� si� bezpo�rednio w bazie, ale pozwala na jej udzielenie tak�e w przypadkach wymagaj�cych po��czenia fakt�w zapisanych w danych ze sob� lub wcze�niej wygenerowan� wiedz�. \emph{Data Minig} znajduje bardzo szerokie zastosowania w wielu dziedzinach, pocz�wszy od handlu elektronicznego, przez analiz� trend�w na rynkach finansowych, po diagnostyk� medyczn�.
%
%Jednym z naukowc�w zajmuj�cych si� dziedzin� Data Miningu jest prof. Ryszard S. Michalski. Wraz z innymi naukowcami z \emph{George Mason University (USA)} stworzy� on system \emph{Vinlen}, kt�rego zadaniem jest realizowanie zada� \emph{Data Miningu}. 
%W systemie \emph{Vinlen} standardowe operatory relacyjne definiowane przez  \emph{SQL} s� wykorzystywane do poprzez utworzenie nowych typ�w operator�w, zwanych \emph{operatorami generowania wiedzy} (ang. \emph{knowledge generation operators (KGO)}). Operatory te dzia�aj� na segmentach wiedzy, stanowi�cych po��czenie jednej lub wi�kszej ilo�ci tabel z relacyjnej bazy danych oraz wiedzy przechowywanej w bazie wiedzy. Operatory te na podstawie kilku segment�w wej�ciowych generuj� nowy segment wyj�ciowy, kt�ry mo�e by� zapisany w bazie wiedzy.
%
%U podstaw tworzenia operator�w generowania wiedzy leg�y dwie zasady:
%\begin{itemize}
%	\item rezultaty ich dzia�ania musz� by� przedstawiane w formie �atwej do zrozumienia przez u�ytkownik�w systemu
%	\item wiedza generowana przy ich u�yciu musi by� wyra�ona w spos�b mo�liwie zwarty i efektywny
%\end{itemize}
%
%Te zasady zosta�y spe�nione poprzez zastosowanie metod dedukcji, w kt�rych spos�b tworzenia opisu danych jest jak najbardziej naturalny dla cz�owieka -- dane te zapisywane s� w j�zyku atrybutywnym. J�zyk atrybutywny to system logiczny, kt�ry ��czy elementy logiki \texttt{klasycznej, rachunku pierwszego rz�du, oraz logiki wielowarto�ciowej??}. S�u�y on zar�wno jako system wnioskowania, jak i jako j�zyk reprezentacji wiedzy.
%
%Regu�y atrybutywne, podstawowy spos�b reprezentacji wiedzy w systemie \emph{Vinlen} s� znaczenie bardziej u�yteczne ni� konwencjonalne regu�y decyzyjne oparte na warunkach postaci <atrybut-relacja-warto��>.


\begin{itemize}
	\item \href{http://www.mli.gmu.edu/papers/2003-2004/Kaufman-Michalski-IIS03-final.pdf}{http://www.mli.gmu.edu/papers/2003-2004/Kaufman-Michalski-IIS03-final.pdf}
	\item \href{http://www.mli.gmu.edu/papers/96-2000/98-5.pdf}{http://www.mli.gmu.edu/papers/96-2000/98-5.pdf}
	
	\item \href{http://www.galaxy.gmu.edu/stats/colloquia/AbstractsFall2004/CollSept17.html}{http://www.galaxy.gmu.edu/stats/colloquia/AbstractsFall2004/CollSept17.html}
\end{itemize}


\end{document}

%%
\end{streszczenie}
%%
%% ======== G��WNA CZʌ� PRACY ========
%%
%% ==== WST�P ====
%%
\begin{wstep}
%%

\section{Wprowadzenie}
\frame{
	\frametitle{Indukcyjne bazy danych}
		\begin{columns}[t]
			\begin{column}{0.5\textwidth}
				\begin{center}
					\includegraphics[width=\textwidth]{img/knowledge_mining.png}
				\end{center}
			\end{column}
			\begin{column}{0.5\textwidth}
				\begin{idatabase}
					\begin{itemize}
						\item Naturalne rozwini�cie system�w bazodanowych
						\item Udzielaj� odpowiedzi\\ nie tylko bezpo�rednio\\ na podstawie danych
						\item Pozwalaj� na zsyntezowanie wiedzy wygenerowanej poprzez indukcyjne wnioskowanie z~fakt�w i~wcze�niejszej wiedzy
    			\end{itemize}
    		\end{idatabase}
    \end{column}
	\end{columns}
}

%%
\end{wstep}
%%
%% ==== ROZDZIA� 1 ====
%%
%% A tutaj tak dla przyk�adu jest \part
\part{Musztarda kotwiczna}

\chapter{Wprowadzenie do przypraw walcowych}
%%
% \input{Mgr_roz1}
%%
\mgrclosechapter
%%
%% ==== ROZDZIA� 2 ====
%%
% \input{Mgr_roz2}
%%
%%
%% ======== DODATKI ========
%%
%%
%% ======== BIBLIOGRAFIA ========
%%
\begin{thebibliography}{99}
\bibitem {bib1} {Michalski, Kaufman, Pietrzykowski, Sniezynski,
    Wojtusiak, Sharma, Seeman, Fischthal, Alkharouf, White, Draminski,
    Glowinski, ''Inductive Databases and Knowledge Scouts''}

\bibitem {bib2} {Kaufman, K. and Michalski, R. S., ''The Development
    of the Inductive Database System VINLEN: A Review of Current
    Research,'' International Intelligent Information Processing and
    Web Mining Conference, Zakopane, Poland, 2003}

\bibitem {bib3} {Michalski, R. S. and Kaufman, K., ''Data Mining and
    Knowledge Discovery: A Review of Issues and a Multistrategy
    Approach,'' Machine Learning and Data Mining: Methods and
    Applications, R. S. Michalski, I. Bratko and M.  Kubat (Eds.), pp.
    71-112, London: John Wiley \& Sons, 1998}

\bibitem {bib4} {R. S. Michalski, ''Knowledge Mining and Inductive
    Databases: An Emerging New Research Direction'', School of
    Computational Sciences, George Mason University, 2004}
    
%\bibitem {weka} Weka -- \href{http://www.cs.waikato.ac.nz/ml/weka}{http://www.cs.waikato.ac.nz/ml/weka}

%\bibitem {yale} YALE -- \href{yale.cs.uni-dortmund.de}{yale.cs.uni-dortmund.de}

\end{thebibliography}
%%
%% ======== DODATKOWE ELEMENTY PRACY (nieobowi�zkowe) ======== 
%%
%\printindex  
%%

\end{document}