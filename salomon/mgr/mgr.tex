%\documentclass{article} % opcje: robocza,man
\documentclass{mgragh} % opcje: robocza,man
\parindent  20pt
\parskip    0.25cm
\textwidth  15.5cm
\textheight 23.5cm
\headheight 16pt
\linespread{1.2}

\usepackage[usenames]{color}
\usepackage{float}
\usepackage[cp1250]{inputenc}  % opcja latin2 dla Linuxa lub cp1250 dla Windows
\usepackage{margins}

\usepackage[polish]{babel}
\usepackage[T1]{fontenc}
\usepackage{polski}
%\usepackage{graphicx}
\usepackage[urlcolor=blue, colorlinks=true, bookmarks=true, bookmarksnumbered=true]{hyperref}
\usepackage{epsfig}
\usepackage{graphics}
\usepackage{listings}
%pakiet do listowania kod�w �r�d�owych

\definecolor{ListingCommentsColor}{rgb}{0,0.5,0}
\definecolor{ListingStringsColor}{rgb}{0,0,0.5}

\lstset{language=java,frame=none,keywordstyle=\color{blue},
	stringstyle=\color{ListingStringsColor},basicstyle=\small,
	commentstyle=\color{ListingCommentsColor},tabsize=2,breaklines,
	breakautoindent=true}
%pakiet do grupowania m.in listing�w na jednej stronie
\usepackage{framed}
% dzielenie wyrazow
\hyphenation{prze-cho-wy-wa-nych dzie-dzi-ny ma-szy-no-we-go wy-po-sa-�e-nia}

%%
%%
\makeindex

%\bibliographystyle{ddabbrv}
\bibliographystyle{plain}
%\bibliographystyle{plplain}
%\nocite{*}

\begin{document}
%%
%%
%% ======== METRYCZKA PRACY ========
\title{Komponentowa architektura indukcyjnej bazy danych jako platformy uczenia
maszynowego}
\author{Nikodem Jura, Krzysztof Rajda}
\promotor{dr hab. in�. Marek Kisiel-Dorohinicki}
\uczelniaNazwa{Akademia G�rniczo-Hutnicza}
\uczelniaImienia{im. Stanis�awa Staszica}
\wydzial{Wydzia�: Elektroniki, Automatyki, Informatyki i Elektrotechniki}
\kierunek{Informatyka}
\specjalnosc{In�ynieria system�w informatycznych i baz danych}
\rok{2007}

\maketitle
%%
%\slowakluczowe{}

%\keywords{a,b,c,d,e,f,g,h} 
%%
%%
%% ======== NASZE MAKRA ========
%%
%% ************ AKADEMIA G/ORNICZO-HUTNICZA W KRAKOWIE *************
%% ***************** Wydzial Matematyki Stosowanej *****************
%% ****************** PRACA MAGISTERSKA w LaTeX-u ******************
%%    autor: ------
%%    Copyright (C) 2002 by ------
%% ********************* Plik z definicjami ************************
%%

%----------------- nasze definicje -----------------% 
\newtheorem{stw}{\indent Stwierdzenie}[chapter]
%------------------------------%
\newcommand{\id}[1]{\index{#1}}  
\newcommand{\wi}[1]{#1\index{#1}}  
\newcommand{\wwi}[1]{\emph{#1}\index{#1}}  
\newcommand{\mwi}[1]{\textbf{#1}\index{#1}}
\newcommand{\ii}[1]{\textit{#1}}
%\newcommand{}{}

%---------------------------------------------------%



%%
%% ======== SPIS TRE�CI ========
%%
\tableofcontents

%%
%% ======== STRESZCZENIE PRACY (POLSKIE) ========
%\begin{streszczenie}
%%
%\documentclass[a4paper,12pt]{article}
\usepackage[polish]{babel}
\usepackage[cp1250]{inputenc}
\usepackage[T1]{fontenc}
\usepackage[dvips]{graphicx}
\usepackage{fancyvrb}

\usepackage[urlcolor=blue, colorlinks=true, bookmarks=true, bookmarksnumbered=true]{hyperref}
\usepackage{indentfirst}
\usepackage{times}
\usepackage{epsfig}
\usepackage{color}

\usepackage{fancyhdr} 

\pagestyle{fancy} 

\rfoot{SALOMON}


%ustawienie wielkosci akapitu
%\setlength{\parindent}{0mm}
%%odstep miedzy akapitami
%\setlength{\parskip}{2mm}

%ustawienia rozmiaru tekstu
%    \textwidth 16cm
%    \textheight 24cm 
%    \topmargin -15mm
%    \evensidemargin-3mm
%    \oddsidemargin -3mm   

\title{\emph{Salomon} extended abstract}

\author{Nikodem Jura \emph{(nico@icslab.agh.edu.pl)}
        \\
        Krzysztof Rajda \emph{(krzycho@student.uci.agh.edu.pl)}
				}                       

\begin{document}

\maketitle
%wylaczenie numerowania na I stronie
\thispagestyle{empty}

\begin{center}
\textcolor{blue}{\href{http://salomon.iisg.agh.edu.pl}{http://salomon.iisg.agh.edu.pl}}
\end{center}

\begin{center}
Prowadz�cy: dr in�. Marek Kisiel-Dorohinicki
\end{center}
\emph{Knowledge Mining} to nowo rozwijana metodologia, kt�ra pozwala wydobywa� u�yteczn� wiedz� zar�wno z danych jak i z wiedzy og�lnej. W metodologii tej bardzo du�y nacisk k�adziony jest na prezentacj� wynik�w w spos�b
przyjazny dla cz�owieka, np. za pomoc� j�zyka zbli�onego do j�zyka naturalnego, oraz za pomoc� r�norodnych graficznych reprezentacji. Pozwala one na generowanie wiedzy zar�wno z wielkich system�w bazodanowych gdzie dane mog� by� niekompletnych a nawet cz�ciowo b��dne.

Jednym z naukowc�w zajmuj�cych si� dziedzin� \emph{Knowledge Mining} jest \emph{prof. Ryszard S. Michalski}. Wraz z innymi naukowcami z \emph{George Mason University (USA)} stworzy� on system \emph{Vinlen}, kt�rego zadaniem jest realizowanie zada� metodologii \emph{Knowledge Mining} wykorzystuj�c indukcyjne bazy danych. Celem tego projektu jest integracja wielu istniej�cych metod wydobywania wiedzy takich jak np. \emph{AQ} (algorytm generacji regu� atrybutowych), czy metod konceptualnego klastrowania. W systemie zosta�y r�wnie� zawarte mechanizmy zarz�dzania danymi, operatory wizualizacji wiedzy (ITG) oraz tworzenie graf�w powi�za�. W celu �atwego u�ytkowania systemu zosta� on wyposa�ony w przyjazny graficzny interfejs u�ytkownika.

Z zwi�zku z pewnymi ograniczeniami architektury \emph{VINLEN} powsta� projekt maj�cy na celu stworzenie dla niej alternatywy pozbawionej tych ogranicze�.

\begin{itemize}
	\item \textbf{Ca�a funkcjonalno�� umieszczona we wtyczkach.}
	Silnik programu powinnen by� tak lekki jak tylko to mo�liwe. Jego zdaniem jest dostarczenie przyjaznego �rodowiska do realizacji funkcjonalno�ci zawartej we wtyczkach.
	
	\item \textbf{Otwarta architektura}
	Wprowadzenie warstwy izoluj�cej wtyczyki od bazy danych. Otwarta architektura pozwala na wygodne rozszerzanie tej warstwy.
	
	\item \textbf{Niezale�no�� od platformy}
	System powinnen by� niezale�ny od platformy. Powinna by� mo�liwo�� pracy w rozproszonychh hetegeronicznym �rodowisku.
	\item \textbf{Mo�liwo�� definowania przep�ywu sterowania}
	\item \textbf{�atwo�� adaptowania funkcjonalno�ci zawartej w systemie \emph{VINLEN}}
	
\end{itemize}

Ekstrakcja wiedzy to proces iteracyjny podzielony na etapy, kt�re mog� tworzy� cykle.

W ka�dym tym etapie wiedza jest poddawana obr�bce. Na ka�dym takim etapie proces odkrywania mo�e by� ukierunkowany zgodnie
z wymaganiami u�ytkownika. Ka�dy etap tworzy odr�bn� ca�o��. Cz�� etap�w jest roz��czna, zatem mog� by� wykonywane wsp�bie�nie. Architektura Salomona pozwala na rozproszenie, a co za tym idzie -- zr�wnoleglenie, wykonywania takich roz��cznych etap�w oraz na synchronizacj� ich wynik�w.

W Salomonie reprezentacj� etapu jest \emph{zadanie} (Task). Ka�de zadanie mo�e posiada� kilka nast�pnik�w i kilka
poprzednik�w. Ujmuj�c rzecz pro�ciej, zadania mog� by� zorganizowane w postaci grafu. 

Zadanie zdefiniowane jest jako tr�jka:
\begin{itemize}
	\item algorytm(dostarczany w postaci wtyczki, czyli \emph{pluginu}))
	\item dane steruj�ce algorytmem
	\item dane wej�ciowe, z kt�rych algorytm b�dzie wydobywa� wiedz�
\end{itemize}

Na wej�ciu lub wyj�ciu ka�dego zadania mog� pojawi� si� dane (w przypadku algorytm�w, kt�re
dokonuj� selekcji/segregacji danych), wiedza lub obie te rzeczy naraz.

Salomon sk�ada si� z dw�ch zasadniczych cz�ci (Rys \ref{fig:Achitecture}). Pierwsz� z nich
stanowi platforma. Tworz� j�:
\begin{itemize}
	\item silnik, kt�rego zadaniem jest zarz�dzanie i~uruchamianie
zada�
	\item magazyn s�u��cy do przechowywania wiedzy
\end{itemize}

Druga cz�� to zbi�r wtyczek, dostarczaj�cych logik� potrzebn� do skonfigurowania, wykonania i~wy�wietlenia rezultat�w
zadania. System jest otwarty - oznacza to, �e jego funkcjonalno�� mo�e by� w �atwy spos�b rozszerzana poprzez dostarczenie nowych wtyczek.


%spis tresci
%\tableofcontents
%\newpage
%VINLEN inductive database system is brie
%y
%reviewed and illustrated by selected results. The goal of research on VINLEN is to
%develop a methodology for deeply integrating a wide range of knowledge generation
%operators with a relational database and a knowledge base. The current system
%has already integrated an AQ learning system for generating attributional rules in
%two modes: theory formation, in which generated rules are consistent and complete
%with regard to data, and pattern discovery, in which generated rules represent strong
%patterns, not necessarily consistent or complete. It also has integrated a conceptual
%clustering module for splitting data into conceptual classes, and providing descrip-
%tions of those classes. Preliminary data management and knowledge visualization
%operators, such as the intelligent target data generator (ITG) and concept asso-
%ciation graph display, have also been integrated. To facilitate an easy interaction
%with the system, a user-oriented visual interface has been implemented. An exam-
%ple of results from applying VINLEN to a medical problem domain is presented to
%illustrate VINLEN knowledge discovery and representation capabilities.



%\newpage
%Dziedzina baz danych znajduje si� obecnie w fazie niespotykanego rozwoju -- staj� si� one nie tylko wszechobecne, ale i~globalnie po��czone. W zwi�zku z tym pojawi�o si� wiele nowych kierunk�w ich rozwoju, jednym z nich jest \emph{Data Mining}.
%
%\emph{Data Mining} to dziedzina nauki, zajmuj�ca si� przetwarzaniem danych zmagazynowanych w bazach danych w celu wydobycia z nich wiedzy. W przeciwie�stwie do konwencjonalnych baz danych, wiedza ta umo�liwia znalezienie odpowiedzi nie tylko na pytania, na kt�re odpowiedzi znajduj� si� bezpo�rednio w bazie, ale pozwala na jej udzielenie tak�e w przypadkach wymagaj�cych po��czenia fakt�w zapisanych w danych ze sob� lub wcze�niej wygenerowan� wiedz�. \emph{Data Minig} znajduje bardzo szerokie zastosowania w wielu dziedzinach, pocz�wszy od handlu elektronicznego, przez analiz� trend�w na rynkach finansowych, po diagnostyk� medyczn�.
%
%Jednym z naukowc�w zajmuj�cych si� dziedzin� Data Miningu jest prof. Ryszard S. Michalski. Wraz z innymi naukowcami z \emph{George Mason University (USA)} stworzy� on system \emph{Vinlen}, kt�rego zadaniem jest realizowanie zada� \emph{Data Miningu}. 
%W systemie \emph{Vinlen} standardowe operatory relacyjne definiowane przez  \emph{SQL} s� wykorzystywane do poprzez utworzenie nowych typ�w operator�w, zwanych \emph{operatorami generowania wiedzy} (ang. \emph{knowledge generation operators (KGO)}). Operatory te dzia�aj� na segmentach wiedzy, stanowi�cych po��czenie jednej lub wi�kszej ilo�ci tabel z relacyjnej bazy danych oraz wiedzy przechowywanej w bazie wiedzy. Operatory te na podstawie kilku segment�w wej�ciowych generuj� nowy segment wyj�ciowy, kt�ry mo�e by� zapisany w bazie wiedzy.
%
%U podstaw tworzenia operator�w generowania wiedzy leg�y dwie zasady:
%\begin{itemize}
%	\item rezultaty ich dzia�ania musz� by� przedstawiane w formie �atwej do zrozumienia przez u�ytkownik�w systemu
%	\item wiedza generowana przy ich u�yciu musi by� wyra�ona w spos�b mo�liwie zwarty i efektywny
%\end{itemize}
%
%Te zasady zosta�y spe�nione poprzez zastosowanie metod dedukcji, w kt�rych spos�b tworzenia opisu danych jest jak najbardziej naturalny dla cz�owieka -- dane te zapisywane s� w j�zyku atrybutywnym. J�zyk atrybutywny to system logiczny, kt�ry ��czy elementy logiki \texttt{klasycznej, rachunku pierwszego rz�du, oraz logiki wielowarto�ciowej??}. S�u�y on zar�wno jako system wnioskowania, jak i jako j�zyk reprezentacji wiedzy.
%
%Regu�y atrybutywne, podstawowy spos�b reprezentacji wiedzy w systemie \emph{Vinlen} s� znaczenie bardziej u�yteczne ni� konwencjonalne regu�y decyzyjne oparte na warunkach postaci <atrybut-relacja-warto��>.


\begin{itemize}
	\item \href{http://www.mli.gmu.edu/papers/2003-2004/Kaufman-Michalski-IIS03-final.pdf}{http://www.mli.gmu.edu/papers/2003-2004/Kaufman-Michalski-IIS03-final.pdf}
	\item \href{http://www.mli.gmu.edu/papers/96-2000/98-5.pdf}{http://www.mli.gmu.edu/papers/96-2000/98-5.pdf}
	
	\item \href{http://www.galaxy.gmu.edu/stats/colloquia/AbstractsFall2004/CollSept17.html}{http://www.galaxy.gmu.edu/stats/colloquia/AbstractsFall2004/CollSept17.html}
\end{itemize}


\end{document}

%%
%\end{streszczenie}
%%
%% ======== G��WNA CZʌ� PRACY ========
%%
%% ==== WST�P ====
%%
\begin{wstep}
%%

\section{Wprowadzenie}
\frame{
	\frametitle{Indukcyjne bazy danych}
		\begin{columns}[t]
			\begin{column}{0.5\textwidth}
				\begin{center}
					\includegraphics[width=\textwidth]{img/knowledge_mining.png}
				\end{center}
			\end{column}
			\begin{column}{0.5\textwidth}
				\begin{idatabase}
					\begin{itemize}
						\item Naturalne rozwini�cie system�w bazodanowych
						\item Udzielaj� odpowiedzi\\ nie tylko bezpo�rednio\\ na podstawie danych
						\item Pozwalaj� na zsyntezowanie wiedzy wygenerowanej poprzez indukcyjne wnioskowanie z~fakt�w i~wcze�niejszej wiedzy
    			\end{itemize}
    		\end{idatabase}
    \end{column}
	\end{columns}
}

%%
\end{wstep}
%%
%% ==== ROZDZIA� 1 ====
%%
%% A tutaj tak dla przyk�adu jest \part
%\part{Wprowadzenie}

%%
\chapter{Uczenie maszynowe}
%\section{Data maining}
% czy mozna pisac takie ksiazkowe luzne wstawki?

%CZY MOZNA TO TLUMACZYC?
%POCZATEK \\
%Human in vitro fertilization involves collecting several eggs from a woman�s
%ovaries, which, after fertilization with partner or donor sperm, produce several
%embryos. Some of these are selected and transferred to the woman�s uterus. The
%problem is to select the �best� embryos to use�the ones that are most likely to
%survive. Selection is based on around 60 recorded features of the embryos�
%characterizing their morphology, oocyte, follicle, and the sperm sample. The
%number of features is sufficiently large that it is difficult for an embryologist to
%assess them all simultaneously and correlate historical data with the crucial
%outcome of whether that embryo did or did not result in a live child. In a
%research project in England, machine learning is being investigated as a technique
%for making the selection, using as training data historical records of
%embryos and their outcome. \\
%KONIEC

\section{Uczenie maszynowe}

%Cichosz (35-36)
Pr�by opracowania algorytm�w ucz�cych si� nie wynikaj� z ch�ci wyeliminowania
projektant�w z proces�w analizy i projektowania system�w komputerowych, a wi�c
klasycznych zagadnie� in�ynierii oprogramowania. Nie mog� one bowiem by�
alternatyw� dla tradycyjnych metodologii tworzenia oprogramowania. Cele, jakie
stawiaj� sobie tw�rcy algorytm�w ucz�cych si� wynikaj� ze z�o�ono�ci niekt�rych
zagadnie� algorytmicznych -- pr�buj� oni w�a�nie za ich pomoc� opisa� te
problemy, dla kt�rych opracowanie poprawnych i pe�nych algorytm�w klasycznych
jest bardzo trudne lub wr�cz niemo�liwe.

Program ucz�cy mo�na wyobrazi� sobie jako abstrakcyjny, ,,parametryzowalny''
algorytm wykonania zadania. Proces uczenia polega na dobraniu, na podstawie
historycznych warto�ci tych ,,parametr�w'', takich warto�ci, by rozwi�zanie
spe�nia�o za�o�enia projektanta. 

,,Parametry'' te mo�na traktowa� jako pewnego rodzaju \emph{wiedz�}.
Nie s� one podawane do algorytmu w spos�b bezpo�redni (a je�li nawet s� podawane
ich pocz�tkowe warto�ci, to s� one najcz�ciej dalekie od oczekiwanych), ale
odkrywane s� przez sam algorytm podczas procesu uczenia si�.
Z tego te� powodu s� one traktowane jako wiedza niepewna i mog�ca wymaga� weryfikacji.	

Wiedza ta mo�e okre�la� zar�wno sekwencje operacji, kt�re program ma wykona�
podczas rozwi�zywania danego problemu, jak i wyb�r spo�r�d r�nych wariant�w
mo�liwych do podj�cia w danym momenecie decyzji.

Wiedza, kt�ra okre�la strategie osi�gania cel�w nazywana jest
\emph{proceduraln�}, natomiast taka, kt�ra opisuje obiekty i zwi�zki mi�dzy nimi
-- \emph{deklaratywn�}.

Algorytmy pozyskiwania i doskonalenia zdobytej wiedzy nazywane s�
\emph{algorytmami uczenia maszynowego}. 

% Cichosz 40
Zastosowanie algorytm�w uczenia maszynowego do rozwi�zania niekt�rych problem�w
mo�e by� tak�e podyktowane czynnikami ekonomicznymi -- czasem bardziej op�aca
si� zastosowa� algorytmy ucz�ce si�, ni� traci� czas na opracowywanie 
skomplikowanych algorytm�w klasycznych, kt�re i tak w pewnych przypadkach mog�
dzia�a� niepoprawnie.

Dla naprawd� du�ych i z�o�onych problem�w trudne jest opracowanie pe�nych i~poprawnych 
algorytm�w je rozwi�zuj�cych. Z takimi przypadkami mamy do czynienia
np. w zagadnieniach dotycz�cych realnie dzia�aj�cych system�w, w kt�rych trzeba
liczy� si� z du�� dynamik� i nieprzewidywalno�ci� �rodowiska, w kt�rym dzia�a program.
Uzyskanie poprawnie dzia�aj�cych algorytm�w pracuj�cych w takich systemach mo�e
okaza� si� bardzo kosztowne: albo ze wzgl�du na czas i �rodki potrzebne do ich
opracowania, albo ze wzgl�du na zasoby u�ywane podczas ich pracy. Zdarza si�
tak�e, ze opracowanie zadowalaj�cych algorytm�w jest wr�cz niemo�liwe.
Wynika to z tego, �e �rodowiska w kt�rych cz�sto musz� dzia�a� takie algorytmy
s� trudne do opisania -- brakuje dla nich modeli teoretycznych, lub te�
uproszczenia, kt�re musia�y zosta� w nich przyj�te, by wog�le mo�liwe by�o
opisanie danego �rodowiska nie pozwalaj� na uzyskanie wystarczaj�co dok�adnie
dzia�aj�cych algorytm�w.

W wielu zastosowaniach systemy informatyczne powinny dzia�a� mo�liwe
autonomicznie i wymaga� znikomej ingerencji ze strony cz�owieka. Przyk�adami
takich system�w mog� by� systemy kontroli poszczeg�lnych instalacji w
nowoczesnych biurowcach, systemy sterowania robotami przemys�owymi czy pojazdami
bezza�ogowymi. Wymagany stopie� autonomii tych system�w jest niemo�liwy do
uzyskania bez wyposa�enia ich w mo�liwo�� adaptacji, czyli przystosowania si� do
zmieniaj�cych si� warunk�w. Systemy posiadaj�ce zdolno�� adaptacji mog� by�
u�ywane w nieprzewidywalnym �rodowisku lub wielu podobnych �rodowiskach.

Jednym z najwa�niejszych zastosowaniem algorytm�w ucz�cych jest nowoczesna
analiza danych, tzw. \emph{data mining}.
Wsp�czesne zbiory danych, kt�re musz� by� poddawane analizie pochodz� np.
z d�ugotrwa�ych pomiar�w i eksperyment�w naukowych. Zawieraj� ogromne ilo�ci
danych, co powoduje, ze nie mog� one by� analizowane w inny spos�b, ni�
automatycznie. Do ich filtrowania i wyszukiwania zale�no�ci mi�dzy nimi
wykorzystuje si� specjalnie opracowane algorytmy, kt�re potrafi� nie tylko
przetwarza� dane aktualne, ale tak�e wnioskowa� na podstawie danych historycznych,
przechowywanych w ogromnych bazach danych, zwanych \emph{hurtowniami danych}.

%Troche przykladow
\section{Problemy klasyfikacyjne i~algorytmy wykorzystuj�ce drzewa decyzyjne}

%An Introduction to~Classification and Regression Tree (CART) Analysis
%Roger J. Lewis, M.D., Ph.D.
%Department of Emergency Medicine
%Harbor-UCLA Medical Center
%Torrance, California, page 2-3

Rozwi�zywanie zagadnie� klasyfikacyjnych to~jedno z~zastosowa� algorytm�w uczenia maszynowego. Zadania klasyfikacyjne polegaj� na~okre�leniu warto�ci jednego z~atrybut�w danych (zwanego atrybutem \emph{decyzyjnym}) na~podstawie warto�ci innych atrybut�w i~pewnych danych historycznych. Dane historyczne pe�ni� rol� danych trenuj�cych -- algorytm ,,uczy si�'' jakie warto�ci przyjmuje atrybut, kt�rego warto�� ma by� predykowana, w~zale�no�ci od warto�ci pozosta�ych atrybut�w. Do~sprawdzenia poprawno�ci predykcji mo�e zosta� wykorzystany zbi�r danych testuj�cych, dla kt�rego znana warto�� atrybutu decyzyjnego por�wnywana jest z~warto�ci� uzyskan� w~wyniku dzia�ania algorytmu.

\subsection{Drzewa decyzyjne}
\label{lab:dec_trees}

Jednym z~przyk�ad�w reprezentacji wiedzy uzyskanej
za pomoc� algorytm�w uczenia maszynowego s� drzewa decyzyjne.

Przyk�adami takich algorytm�w s� \emph{ID3} i~\emph{C4.5}, wprowadzone  przez \emph{J. Ross Quinlan'a} do~tworzenia na~podstawie danych indukcyjnych modeli klasyfikuj�cych, zwanych  drzewami decyzyjnymi.

Jako wej�cie otrzymuj� one zbi�r rekord�w o~jednakowej strukturze, sk�adaj�cy si� z~pewnej liczby par atrybut-warto��. Jeden z~tych atrybut�w reprezentuje
tzw. kategori� rekordu, jest te� okre�lany mianem atrybutu decyzyjnego.

Celem algorytm�w jest stworzenie drzewa, kt�re na~podstawie warto�ci
pozosta�ych atrybut�w okre�li warto�� atrybutu decyzyjnego.
Zwykle atrybut decyzyjny przyjmuje warto�ci logiczne \emph{(prawda, fa�sz)}, 
\emph{(sukces/pora�ka)} lub podobne. W~ka�dym przypadku jedna z~warto�ci
atrybutu oznacza pora�k�.

Drzewa decyzyjne zawdzi�czaj� sw� nazw� temu, �e prezentowane s� w~formie
struktury drzewiastej (Rys \ref{fig:sample_tree}). Graficzna reprezentacja powoduje,
�e zrozumienie go, czy te� por�wnanie z~innym drzewem jest proste nawet dla cz�owieka. 
Drzewa decyzyjne sta�y si� bardzo popularne w~technikach eksploracji danych (szerzej opisanych w~rozdziale \ref{lab:data_mining}). 
S� one najpowszechniej u�ywane do~klasyfikacji (predykuj� do~kt�rej grupy nale�y dany przypadek), lecz mog� by� r�wnie� u�ywane przy regresji (predykuj� warto��).

% The decision tree method encompasses a~number of specific algorithms, including
% Classification and Regression Trees (CART), Chi-squared Automatic Interaction
% Detection (CHAID), C4.5 and C5.0 (from work by J. Ross Quinlan of Rulequest
% Research Pty Ltd, in St. Ives, Australia, www.rulequest.com).

Metody drzew decyzyjnych obejmuj� wiele algorytm�w. Mi�dzy innymi drzewa
klasyfikuj�ce i~regresyjne (ang. \emph{Classification and Regression Trees
(CART)}), \emph{Chi-squared Automatic
Interaction Detection (CHAID)}, \emph{C4.5} and \emph{C5.0}.

% Decision trees graphically display the relationships found in data. Most
% products also translate the tree-to-text rules such as If Income = High and
% Years on job > 5 Then Credit risk = Good. In fact, decision tree algorithms are
% very similar to~rule induction algorithms which produce rule sets without a
% decision tree.

Drzewa decyzyjne w~graficzny spos�b obrazuj� zale�no�ci wyst�puj�ce w~danych.
Mo�na je r�wnie� przedstawia� w~postaci regu� tekstowych np.
\\
\\
\emph{Je�li Doch�d = Wysoki i~Lata Pracy $>$ 5 Wtedy Ryzyko kredytowe = Ma�e}.
\\
\\
Algorytmy drzew decyzyjnych s� bardzo podobne do~algorytm�w indukcji regu�, kt�re
produkuj� zbiory regu� bez drzew decyzyjnych.

% The primary output of a~decision tree algorithm is the tree itself. The 
% training process that creates the decision tree is usually called induction. 
% Induction requires a~small number of passes (generally far fewer than 100) 
% through the training dataset. This makes the algorithm somewhat less efficient 
% than Na�ve-Bayes algorithms (See Na�ve-Bayes and Nearest Neighbor.), which 
% require only one pass, but significantly more efficient than neural nets, which 
% typically require a~large number of passes, sometimes numbering in the 
% thousands. To~be more precise, the number of passes required to~build a~
% decision tree is no more than the number of levels in the tree. There is no 
% predetermined limit to~the number of levels, although the complexity of the 
% tree as measured by the depth and breadth of the tree generally increases as 
% the number of independent variables increases.

G��wny rezultatem algorytmu drzew decyzyjnych jest w�a�nie drzewo. Proces
treningu, kt�ry tworzy drzewo, jest zwyczajowy nazywany indukcj�. Indukcja wymaga kilku
przej�� (generalnie znacznie mniej ni� 100) przez zbi�r trenuj�cy.
Powoduje to, �e algorytmy te s� mniej wydajne ni� algorytmy 
\emph{Na�ve-Bayes}, kt�re wymagaj� tylko jednego przej�cia, ale znacznie bardziej wydajne
ni� sieci neuronowe, kt�re zazwyczaj wymagaj� wielkiej ilo�ci przej��, czasami
liczonych w~tysi�cach. Dok�adniej ilo�� potrzebnych przej�� wymaganych do
zbudowania drzewa decyzyjnego jest nie wi�ksza ni� wysoko�� drzewa (ilo��
warstw). Nie istnieje okre�lona z~g�ry maksymalna wysoko�� drzewa, jednak�e
z�o�ono�� drzewa mierzona jako jego wysoko�� i~szeroko�� generalnie ro�nie je�li
ro�nie ilo�� niezale�nych zmiennych.


%\usepackage{graphics} is needed for \includegraphics
\begin{figure}[htp]
\begin{center}
  \includegraphics[width=0.3\textwidth]{img/sample_tree.jpg}
  \caption[labelInTOC]{Drzewo decyzyjne}
  \label{fig:sample_tree}
\end{center}
\end{figure}

\section{Drzewa decyzyjne}
% http://www.dbmsmag.com/9807m05.html
% When a businessperson needs to make a decision based on several factors, a
% decision tree can help identify which factors to consider and how each factor
% has historically been associated with different outcomes of the decision. For
% example, in our credit risk case study (See the sidebar Predicting Credit Risk),
% we have data for each applicant�s debt, income, and marital status. A decision
% tree creates a model as either a graphical tree or a set of text rules that can
% predict (classify) each applicant as a good or bad credit risk.

Kiedy biznesmen potrzebuje podj�� decyzje bazuj�c na kilku wsp�czynnikach,
drzewa decyzyjne mog� pomoc w wybraniu odpowiednich wsp�czynnik�w oraz 
opowiedzie� jak dany wsp�czynnik w przesz�o�ci wp�ywa� na decyzje.
Przyk�adowo, podczas obliczania ryzyka kredytowego, posiadamy dla ka�dego
potencjalnego kredytobiorcy dane o jego kredytach, przychodach oraz jego
statusie materialnym. Drzewa decyzyjne tworz� zar�wno drzewo w postaci
graficznej, jak zbi�r regu� tekstowych, dzi�ki kt�rym mo�emy okre�li�,
czy dana aplikacja jest obarczona du�ym lub ma�ym ryzykiem kredytowym.

% A decision tree is a model that is both predictive and descriptive. It is called
% a decision tree because the resulting model is presented in the form of a tree
% structure. (See Figure 1.) The visual presentation makes the decision tree model
% very easy to understand and assimilate. As a result, the decision tree has
% become a very popular data mining technique. Decision trees are most commonly
% used for classification (predicting what group a case belongs to), but can also
% be used for regression (predicting a specific value).

Drzewo decyzyjne jest modelem, kt�ry jest zar�wno predyktywnym jak i opisowym.
%!!!predictive and descriptive!!!.
Nazywany jest drzewem decyzyjnym, poniewa� prezentowany jest w formie
drzewiastej struktury~\ref{fig:sample_tree}. Graficzna reprezentacja powoduje,
�e zrozumienie go czy por�wnanie go nie jest trudnym zadaniem. Drzewa decyzyjne
sta�y si� bardzo popularne w technikach ,,data maning''. Drzewa decyzyjne s�
najpowszechniej u�ywane do klasyfikacji (predykuj� do kt�rej grupy nale�y dany
przypadek), lecz mog� by� r�wnie� u�ywane przy regresji (predykuj� warto��).

% The decision tree method encompasses a number of specific algorithms, including
% Classification and Regression Trees (CART), Chi-squared Automatic Interaction
% Detection (CHAID), C4.5 and C5.0 (from work by J. Ross Quinlan of Rulequest
% Research Pty Ltd, in St. Ives, Australia, www.rulequest.com).

Metody drzew decyzyjnych obejmuj� wiele algorytm�w. Mi�dzy innymi drzewa
klasyfikuj�ce i regresyjne (ang. \emph{Classification and Regression Trees
(CART)}), \emph{Chi-squared Automatic
Interaction Detection (CHAID)}, \emph{C4.5} and \emph{C5.0}.

% Decision trees graphically display the relationships found in data. Most
% products also translate the tree-to-text rules such as If Income = High and
% Years on job > 5 Then Credit risk = Good. In fact, decision tree algorithms are
% very similar to rule induction algorithms which produce rule sets without a
% decision tree.

Drzewa decyzyjne w graficzny spos�b obrazuj� zale�no�ci wyst�puj�ce w danych.
Mo�na je r�wnie� przedstawia� w postaci regu� tekstowych np.
\\
\\
\emph{Je�li Doch�d = Wysoki i Lata Pracy $>$ 5 Wtedy Ryzyko kredytowe = Ma�e}.
\\
\\
Algorytmy drzew decyzyjnych s� bardzo podobne do algorytm�w indukcji regu�, kt�re
produkuj� zbiory regu� bez drzew decyzyjnych.

% The primary output of a decision tree algorithm is the tree itself. The 
% training process that creates the decision tree is usually called induction. 
% Induction requires a small number of passes (generally far fewer than 100) 
% through the training dataset. This makes the algorithm somewhat less efficient 
% than Na�ve-Bayes algorithms (See Na�ve-Bayes and Nearest Neighbor.), which 
% require only one pass, but significantly more efficient than neural nets, which 
% typically require a large number of passes, sometimes numbering in the 
% thousands. To be more precise, the number of passes required to build a 
% decision tree is no more than the number of levels in the tree. There is no 
% predetermined limit to the number of levels, although the complexity of the 
% tree as measured by the depth and breadth of the tree generally increases as 
% the number of independent variables increases.

G��wny rezultatem algorytmu drzew decyzyjnych jest w�a�nie drzewo. Proces
treningu tworzy drzewo, jest zwyczajowy nazywany indukcj�. Indukcja wymaga kilku
przej�� (generalnie znacznie mniej ni� 100) przez zbioru trenuj�cy (eg.
training dataset). Powoduje to, �e algorytmy te s� mniej wydajne ni� algorytmy 
\emph{Na�ve-Bayes}, kt�re wymagaj� tylko jednego przej�cia, ale znacznie wydajne
ni� sieci neuronowe, kt�re zazwyczaj wymaj� wielkiej ilo�ci przej��, czasami
liczonych w tysi�cach. Dok�adniej ilo�� ilo�� potrzebnych przej�� wymaganych do
zbudowania drzewa decyzyjnego jest nie wi�ksza ni� wysoko�� drzewa (ilo��
warstw). Nie istnieje okre�lona z g�ry maksymalna wysoko�� drzewa, jednak�e
z�o�ono�� drzewa mierzona jako jego wysoko�� i szeroko�� generalnie ro�nie je�li
ro�nie ilo�� niezale�nych zmiennych.


%\usepackage{graphics} is needed for \includegraphics
\begin{figure}[htp]
\begin{center}
  \includegraphics[width=0.3\textwidth]{img/sample_tree.jpg}
  \caption[labelInTOC]{Drzewo decyzyjne}
  \label{fig:sample_tree}
\end{center}
\end{figure}

\subsection{Om�wienie algorytm�w}

\subsubsection{ID3}

\subsubsection{C4.5}

\section{Darmowe oprogramowanie do uczenia maszynowego}

%http://www.thearling.com/text/dmwhite/dmwhite.htm
%TODO: jest sekcja w tym dokumencie o narzedziach
Wykorzystanie data miningu u�atwia �atwy dost�p do bardzo dobrych narz�dzi, jak
cho�by przedstawione poni�ej \emph{Weka} (\ref{lab:weka}) i \emph{YALE}
(\ref{lab:yale}).

\subsection{Weka}
\label{lab:weka}

\emph{Weka} to narz�dzie do prowadzenia eksperyment�w
zwi�zanych z~uczeniem maszynowym stworzone w \emph{University of
Waikato}~\cite{weka}. Narz�dzie to znajduje szerokie zastosowanie
w dydaktyce i badaniach naukowych. Mo�e r�wnie� udost�pnia� swoj�
funkcjonalno�� innym programom. Stanowi ono zbi�r narz�dzi do
zarz�dzania danymi w r�norodnych formatach, algorytm�w uczenia
maszynowego i~oceniaj�cych uzyskiwane rozwi�zania a~tak�e
�rodowisko do por�wnywania r�nych algorytm�w. �rodowisko zapewnia
przyjazny interfejs u�ytkownika, co czyni je �atwym
w~wykorzystaniu.

\emph{Weka} pozwala na wykorzystywanie r�norodnych �r�de� danych,
pocz�wszy od plik�w tekstowych w wielu formatach, poprzez
relacyjne bazy danych po dane umieszczone w Internecie. Dane te
przetwarzane s� przez tzw. \emph{filtry}, kt�re pozwalaj� na
wyodr�bnienie z nich potrzebnych informacji i przedstawienie ich w
czytelnej postaci.

Najwi�ksz� grup� algorytm�w zaimplementowanych w programie
\emph{Weka} s� algorytmy klasyfikuj�ce. Wykorzystuj� one wiele
form reprezentacji nauczonej wiedzy: od drzew i~regu� decyzyjnych,
przez maszyny wektorowe po sieci Bayesowskie.

Inn� grup� algorytm�w s� algorytmy grupuj�ce (klastruj�ce). Bazuj�
one na schematach \emph{k-Means}, \emph{EM}, \emph{Cobweb},
\emph{X-means}, \emph{FarthestFirst}. Utworzone za ich pomoc�
klastry mog� by� wizualizowane oraz por�wnywane do wzorcowych,
je�li takie s� do dyspozycji.

System \emph{Weka} implementuje r�wnie� popularny algorytm
\emph{Apriori}, wykorzystywany do znajdowania regu� asocjacyjnych.
Dzia�a on tylko na danych dyskretnych i pozwala na identyfikacj�
zale�no�ci statystycznych mi�dzy r�nymi grupami atrybut�w.

\emph{Weka} dostarcza przyjazny graficzny interfejs u�ytkownika,
kt�ry jest bardzo u�yteczny w praktyce.
Pozwala on nie tylko na sprawne wczytywanie danych do systemu,
ale tak�e na ich wizualizacj� w formie czytelnych diagram�w i~wykres�w.

Jedn� z ciekawszych funkcji systemu jest mo�liwo�� graficznego
definiowania przep�ywu danych i sposobu ich przetwarzania. �r�d�a
danych, filtry, klasyfikatory i algorytmy oceniaj�ce mog� by�
��czone graficznie za pomoc� linii. W~ten spos�b mo�na okre�la�
jaki ma by� przep�yw danych w systemie. Konfiguracja konkretnych
danych i metod ich przetwarzania mo�e by� zapisania i za�adowana
ponownie do powt�rnego wykorzystania.


\subsection{YALE}
\label{lab:yale}
\emph{YALE} (Yet Another Learing Environment), podobnie jak
opisywana powy�ej \emph{Weka}, jest systemem do prowadzenia
eksperyment�w zwi�zanych z uczeniem maszynowym.

Jego g��wn� zalet� jest bogaty zestaw operator�w, kt�re
umo�liwiaj� tworzenie ,,�a�cuch�w'' z�o�onych z r�nych problem�w
zwi�zanych z~uczeniem. Dane, na kt�rych operuj� operatory s� dla
nich ca�kowicie prze�roczyste i przechowywane mog� by� w
r�norodnych formatach -- za odpowiednie transformacje pomi�dzy
nimi odpowiada samo �rodowisko.

Dzi�ki operatorom i mo�liwo�ci ich �atwego ��czenia przy u�yciu
graficznego interfejsu u�ytkownika, a tak�e dzi�ki j�zykowi
skryptowemu bazuj�cemu na \emph{XML}, \emph{YALE} jest cz�sto
u�ywane do prowadzenia bada� i~eksperyment�w z r�norodnymi
algorytmami uczenia maszynowego.

\emph{YALE} dostarcza ponad 200 operator�w, jak cho�by algorytmy
klasyfikuj�ce, asocjacyjne, klastruj�ce, wyszukuj�ce regu�y czy
oparte na drzewach decyzyjnych. Dodatkowo mo�liwe jest
wykorzystanie wielu algorytm�w dostarczanych przez �rodowisko
\emph{Weka} w spos�b analogiczny do operator�w wbudowanych. Przed
przetworzeniem dane mog� by� poddane dyskretyzacji, filtrowaniu
czy normalizacji, a po zako�czeniu oblicze� odpowiedniej
walidacji.

\emph{YALE} pozwala na eksperymentowanie i~poszukiwanie najlepszego dla danego problemu
po��czenia r�nych metod uczenia maszynowego. Dzi�ki modularnej strukturze operator�w
mo�liwa jest wymiana pojedynczych ogniw ,,�a�cuch�w'' oblicze� w celu uzyskania
najlepszego efektu. Dodatkowo ,,�a�cuchy'' te mog� by� zagnie�d�ane, tworz�c skomplikowane
drzewa operator�w.
\section{Indukcyjne bazy danych}

% przetlumaczone z pdf-a
Indukcyjne bazy danych �ci�le integruj� bazy danych z konepcj� \emph{data mining}.
G��wn� ich ide� stanowi to, �e zar�wno dane jak i wzorce s� przetwarzane 
w ten sam spos�b, a indukcyjny j�zyk zapyta� pozwala u�ytkownikowi 
na zadawanie zapyta� i manipulacj� danymi wzorcami.

Od pocz�tku istnienia idei data miningu zdawano sobie spraw�,
�e proces ich przetwarzania powinien by� wspierany przez technologi� baz danych.
W ostatnich latach idea ta zosta�a sformalizowana jako koncepcja
\emph{indukcyjnych baz danych}. 

%cytat z artykulu
Integracja technologii baz danych z nowoczesnymi metodami
indukcyjnego generowania wiedzy jest naturalnym kierunkiem rozwoju
system�w bazodanowych. Systemy nazywane indukcyjnymi bazami
danych potrafi� odpowiedzie� nie tylko na pytania, dla kt�rych
odpowied� znajduje si� w bazie danych, ale r�wnie� na pytania,
kt�re wymagaj� zsyntetyzowania i zastosowania wiedzy,
wygenerowanej przez indukcyjne wnioskowanie z fakt�w z bazy danych
i wcze�niejszej wiedzy~\cite{bib3}. Schemat typowej indukcyjnej
bazy danych przedstawiony jest na rysunku~\ref{fig:inddb}.

% \begin{figure}[!ht]
%     \centering
%         \includegraphics[width=0.90\textwidth]{img/knowledge_mining.png}
%     \caption{Indukcyjna baza danych~\cite{bib2}}
%     \label{fig:inddb}
% \end{figure}

Jak zosta�o ju� wy�ej wspomniane, indukcyjne bazy danych przechowuj� nie dane
ale r�wnie� wzorce. Mo�na za�o�y�, �e zar�wno dane, jak i wzorce stanowi� zbiory zbior�w.
To za�o�enie wynika z analogii do tradycyjnych, relacyjnych baz danych. 
Relacyjna baza danych zawiera zbi�r relacji, kt�re to stanowi� zbiory \emph{tupli}, 
a wi�c mo�na powiedzie�, ze stanowi zbi�r zbior�w.

W przypadku baz indukcyjnych rozr�nia si� natomiast zbi�r trenuj�cy od testuj�cego, 
zbi�r poprawnie zaklasyfikowanych przyk�ad�w od zaklasyfikowanych b��dnie itp.
 
To samo, co tyczy si� danych, odnosi si� r�wnie� do wzorc�w.
I rzeczywi�cie -- podczas procesu przetwarzania wiedzy mo�na pracowa� 
na r�nych zbiorach wzorc�w.  Zbiory te mog� odnosi� si� do odmiennych hipotez 
stworzonych na podstawie r�nych zbior�w danych,
czy r�nych ustawie� parametr�w algorytm�w je tworz�cych.

%jezyk (z pdf-a)
Jednym z zasadniczych powod�w, kt�re przyczyni�y si� do sukcesu relacyjnych baz danych
jest opracowanie uniwersalnego j�zyka zapyta�, kt�ry podobnie jak sama algebra relacyjna
dostarcza du�ych mo�liwo�ci przy jego relatywnej prostocie.

Podobnymi cechami powinien wyr�nia� si� j�zyk zapyta� do indukcyjnych baz danych.

Rezultatem zapytania wykonanego w takim j�zyku jest albo zbi�r wzorc�w, albo zbi�r danych.
Jest to tak zwana \emph{w�asno�� zamkni�cia} (\emph{ ang. closure property}).
Stanowi ona analogi� do rezultat�w zapytania do relacyjnych baz danych, gdzie jest nim zawsze relacja.
Rozr�nianie mi�dzy zbiorami wzorc�w i zbiorami danych powoduje, �e potrzebne s� dwa rodzaje zapyta�:
te kt�re generuj� zbiory wzorc�w i zbiory danych. Zapytania do jednocze�nie obu typ�w zbior�w nazywane
s� czasami \emph{zapytaniami krzy�uj�cymi} (\emph{ang. cross-over operations}).

%dodawanie/usuwanie data setow, patternow? (patternow trudniej, nie wiem dokladnie jak, olac na razie)

%wnioskowanie
% z pdf-a
Teoria indukcyjnych baz danych mo�e by� u�yteczna tylko w przypadku,
gdy oferuje mo�liwo�� wnioskowania z danych i zgromadzonej wiedzy.
Na podstawie otrzymanej informacji trenuj�cej indukcyjna baza danych generuje now� lub 
ulepsza wcze�niej posiadan� wiedz�, w pewien ustalony spos�b reprezentowan� 
i przeznaczon� do wykorzystania przy wykonywaniu okre�lonego zadania.
Mechanizm, zgodnie z kt�rym dokonuje si� nabywanie lub doskonalenie wiedzy, 
jest najcz�ciej jednoznacznie wyznaczany przez metod� reprezentacji 
wiedzy oraz posta� informacji trenuj�cej.




%%
%% ==== ROZDZIA� 2 ====
%%
\chapter{Architektura systemu}
\section{Powody powstania Salomona}

%Do czego ma byc to system?
%Istnieja inne...

%1. Struktura rozdzialu 2-giego powinna wygladac tak:
%- do czego ma byc system? (krotko, kilka zdan)
%- jaki ma byc? Czyli te akapity z wymaganiami, ale w innej kolejnosci:
%* ergonomia (prosty, przyjazny interfejs, API)
%* wlasny  sposob reprezentacji wiedzy (m.in. datasety wielowymiarowe)
%* wykorzystanie gotowych algorytmow (np. z Weki) i arch. komponentowa
%* rownoleglosc i rozproszenie
%* bezpieczenstwo
%* licencja
 
%Na koniec dochodzimy do wniosku, ze system powinien miec np. architekture komponenowa, napisany w Javie czy cos takiego.
 
%Kazdy z tych punkcikow oznaczonych '*' powinien miec mniej wiecej taka strukture:
%- jakiej cechy zadamy? (czyli np. "Jednym z postawowych zalozen stawianych systemowi jest prosty i przyjazny interfejs uzytkownika...")
%- dlaczego? ("Przyjazny interfejs ulatwia uzycie aplikacji zwlaszcza mniej doswiadczonym uzytkownikom...")
%- ewentualnie w zestawieniu z istniejacymi systami (Weka ma fajny, Vinlen niewygodny czy cos takiego)

%TODO: Powody powstania systemu
Koncepcja zak�ada stworzenie przyjaznego �rodowiska pozwalaj�cego na odkrywanie i przechowywanie wiedzy
oraz uruchamianie r�nych algorytm�w machine learningu.

Na rynku dost�pne s� r�ne narz�dzia wspomagaj�ce prowadzenie oblicze� zwi�zanych
z zagadnieniami uczenia maszynowego, jak cho�by wy�ej Weka (\ref{lab:weka} ) i Vinlen (\ref{lab:vinlen}).
Analiza ich mo�liwo�ci prowadzi do stwierdzenia, �e posiadaj� one pewne ograniczenia,
utrudniaj�ce ich wykorzystanie do pewnych zastosowa�.
Formu�owane poni�ej wymagania stanowi� cz�sto odpowied� na ograniczenia Weki i Vinlena.

%-gui
Jednym z podstawowych za�o�e� stawianych systemowi jest prosty i przyjazny \textbf{interfejs u�ytkownika}.
Cz�sto cech� decyduj�c� o rozpocz�ciu u�ywania danej aplikacji jest pierwszy kontakt z nia.
Dlatego wa�ne jest, aby graficzny interfejs u�ytkownika
u�atwia� u�ycie aplikacji mniej do�wiadczonym u�ytkownikom i aby nowy u�ytkownik 
systemu nie zosta� zniech�cony ju� na samym pocz�tku.
Nale�y zatem zwr�ci� uwag�, by aplikacja by�a prosta w obs�udze zar�wno dla tw�rcy wtyczek
jak i osoby wykonuj�cej eksperymenty z wykorzystaniem gotowych wtyczek.
System powinien te� by� dostarczany z przyk�adowym zestawem wtyczek i danych, 
kt�re pozwol� na �atwe u�ycie i przetestowanie ka�dego z aspekt�w platformy.

%-api
Opr�cz przyjaznego graficznego interfejsu u�ytkownika platforma powinna
tak�e dostarcza� prostego i dobrze zdefiniowanego \textbf{interfejsu programistycznego}
(ang. Application Programming Interface, \emph{API}).
Pozwoli to na ukrycie przed tw�rcami wtyczek takich aspekt�w implementacyjnych
jak spos�b reprezentacji drzew decyzyjnych czy zapis parametr�w wej�ciowych i rezultat�w wykonania zada�.
Wynika to z tego, �e osoby, kt�re zajmuj� si� tworzeniem wtyczek 
nie powinny musie� bra� pod uwag� sposobu implementacji
platformy, czyli np. przechowywania i dost�pu do danych czy kwesti komunikacji
mi�dzy poszczeg�lnymi zadaniami.

Inn� u�yteczn� w�a�ciwo�ci�, kt�r� nale�y rozwa�y� jest 
umo�liwienie tworzenia wtyczek za pomoc� \textbf{j�zyk�w skryptowych}. Jest to 
szczeg�lnie po�yteczna w�a�ciwo�� np. w przypadku tworzenia nowego algorytmu.
W takiej sytuacji u�yteczna jest mo�liwo�� szybkiego wprowadzania zmian
i natychmiastowego przetestowania zachowania algorytmu bez koniczno�ci
prze�adowywania ca�ej wtyczki.

Wyspecyfikowanie dobrze zdefiniowanych interfejs�w pozwalaj�cych na manipulowanie wiedz�
pozwoli na znaczne uproszczenie procesu tworzenia wtyczek, a dodatkowo pozwoli na 
stworzenie wydajnego i dobrze przetestowanego API.

Z kolei ukrycie sposobu zapisu danych i wiedzy umo�liwi \textbf{�atwiejsz� 
komunikacj�} pomi�dzy r�nymi wtyczkami czy podsystemami. Odizolowanie 
algorytm�w (wtyczek) od zapisu/odczytu danych pozwoli r�wnie� zapewni� 
odpowiedni poziom bezpiecze�stwa. Przyk�adowo system powinien posiada� API 
pozwalaj�ce na tworzenie, manipulacje, odczyt i zapis drzew decyzyjnych, dzi�ki 
czemu r�ne aspekty pracy na drzewach decyzyjnych mo�na by by�o umie�ci� w 
osobnych wtyczkach, niekoniecznie dostarczanych przez tego samego dostawce. 
Jednak przede wszystkim praca nad jedn� cz�ci� (np. prezentacj� drzew) nie wymaga�oby 
wiedzy na temat innych wtyczek.

% - wiedza
\label{lab:knowledge}
System wspomagaj�cy procesy machine learningu,
powinien odznacza� si� sp�jnym i przejrzystym, a jednocze�nie wydajnym sposobem \textbf{reprezentacji wiedzy}.
Wi�kszo�� algorytm�w machine learning wymaga danych wej�ciowych.
Rozr�niamy dane trenuj�ce, kt�re s� wykorzystywane do uczenia algorytmu oraz 
dane testuj�ce. Zbi�r danych testuj�cych wykorzystywany jest do oceny wynik�w 
algorytmu. Oba zbiory zazwyczaj stanowi� losowy podzbi�r dost�pnych danych.

W znanych nam systemach u�ywan� s� jednowymiarowe zbiory trenuj�ce (jedna tabela, 
bez powi�za� z innymi). Taki model pasuje do wi�kszo�ci algorytm�w ucz�cych, jednak 
istnieje pewne ryzyko utraty danych, kt�re nios� z sob� relacje mi�dzy danymi. 
Dlatego chcieliby�my, aby system tworzy� i operowa� na wielowymiarowych zbiorach 
trenuj�cych. Z punktu widzenie algorytmu struktura takiego zbioru nie powinna 
odbiega� od wej�ciowej bazy danych. Jedyna r�nica powinna polega� na ilo�ci 
dost�pnych danych. Takie te� jest pierwsze i najbardziej naturalne zrozumienie 
zbioru trenuj�cego, jako podzbioru danych o niezmienionej strukturze.

System powinien tworzy� podzbi�r danych przed utworzeniem atrybut�w.
Jest to podej�cie r�ne o tego, kt�re stosuje si� w innych aplikacjach.
Przyk�adowo w Wece zbi�r danych jest pozdbiorem zbioru atrybut�w. 
W proponowanym podej�ciu zbi�r danych nie b�dzie wymaga� kopiowania danych 
z zewn�trznego �r�d��, a jedynie definiowa� przepis na utworzenie takowego 
(np. w postaci zapytania, stanowi�cego zbi�r warunk�w).
Z kolej zbi�r atrybut�w stanowi� b�dzie opis transformacji oryginalnych danych
na dane wej�ciowe do algorytm�w ucz�cych. Wyr�nienie wartwy atrybut�w jest konieczne,
gdy� wi�kszo�� algorytm�w wymaga na wej�ciu przetworzonia danych.
Przetworzenie polega g��wnie na okre�leniu typu danego atrybutu, czyli np. czy jest to atrybut wej�ciowy
czy wyj�ciowy oraz jakiego jest rodzaju (napisowy, liczbowy, wyliczeniowy).

Rozwi�zanie przyj�te w systemie zapewnia szereg korzy�ci. Atrybuty budowane na zbiorach danych umo�liwi�
zachowanie oryginalnej, cz�sto wielowymiarowej struktury oryginalnych danych i zwi�zk�w mi�dzy nimi.
Zastosowanie warstwy po�redniej mi�dzy atrybutami a danymi, w postaci zbior�w danych, ma te� inne zalety.
Poniewa� zbiory danych stanowi� jedynie opis, w jakis spos�b oryginalne dane mog� zosta� wydobyte
i jak maj� by� interpretowane, pozwala to na przechowywanie tych danych w zewn�trznych �r�d�ach.
Dzi�ki temu nie trzeba ich kopiowa� do bazy, na kt�rej maj� by� wykonywane algorytmy,
co wi��e si� z nie tylko z ich bezpiecze�stwem, ale tak�e pozwala algorytmom na prac� z zawsze aktualnymi
danymi bez konieczno�ci dodatkowej synchronizacji mi�dzy �r�d�ami danych, a aplikacj�, kt�ra z nich korzysta.
Nie bez znaczenia jest te� fakt, �e przechowuj�c dane w zewn�trznych �r�d�ach danych
i unikaj�c konieczno�ci ich kopiowania mo�na oszcz�dzi� sporo przestrzeni dyskowej,
co mo�e mie� du�e znaczenie przy pracy z ogromnymi zestawami danych.

% - architektura komponentowa
Jednym z podstawowych wymaga� stawianym systemowi, jest \textbf{architektura komponentowa}.
W celu poprawnego dzia�ania w wielkim rozproszonym 
�rodowisku, platforma uruchomieniowa powinna by� dobrze odseparowana od samych 
algorytm�w, co umo�liwi wsp�prac� w jednym �rodowisku r�nych wersji 
systemu, a nawet r�nych jego implementacji. Budowa systemu powinna w jak 
najwi�kszym stopniu wspiera� mo�liwo�� powstania r�nych niezale�nych 
implementacji algorytm�w uczenia maszynowego, dostarczanych przez r�nych 
dostawc�w. Dzi�ki odseparowaniu wtyczek od plaftormy i przeniesieniu na ni� 
odpowiedzialno�ci za zapewnienie komunikacji mi�dzy poszczeg�lnymi wtyczkami
mo�liwe b�dzie np. zastosowanie oblicze� r�wnoleg�ych i rozproszonych, 
bez dodatkowych wymaga� stawianych wtyczkom.

Przeniesienie ci�aru oblicze� z plaftormy do wtyczek poci�ga za sob� konieczno��
zapewnienia przez platform� mechanizm�w aktualizacji wtyczek. Mog� by� one umieszczane
w zdalnych lokalizacjach, tak wi�c platforma powinna posiada� mechanizmy kontroluj�ce
wersje u�ywanych wtyczek i w razie potrzeby aktualizuj�ce je
w spos�b mo�liwie wygodny dla u�ytkownika.

% gotowe algorytmy
Architektura komponentowa i niezale�no�� pomi�dzy poszczeg�lnymi wtyczkami 
umo�liwi \textbf{wykorzystanie gotowych algorytm�w}. W Vinlenie zosta�y cz�ciowo 
wykorzystane implementacje algorytm�w przygotowanych do wcze�niejszych 
projekt�w jak np. program Emerald (przypisek). Znajdziemy tam takie algorytmy jak 
Abacuse, AQ czy Cluster. Cz�� z tych implementacji jest bardzo nowatorska i dysponuje 
wielkimi mo�liwo�ciami. W Wece mo�emy znale�� jeszcze wi�ksz� gam� dost�pnych 
algorytm�w. Dlatego te� system powinien charakteryzowa� si� odpowiednia 
architektur�, kt�ra w �atwy spos�b pozwoli na powt�rne u�ycie ju� 
zaimplementowanych algorytm�w.

W przypadku u�ycia algorytm�w pochodz�cych z r�nych, nie zawsze pewnych, �r�de�
pojawia si� problem bezpiecze�stwa oraz zaufania do r�nych 
implementacji. Naturalnym rozwi�zaniem tego problemu wydaj� si� zastosowanie 
bezpiecznych j�zyk�w programowania, takich jak Java wraz z jej systemem bezpiecze�stwa
i ograniczenia uprawnie�.

% - agenty
System powinien reagowa� na zmiany w �rodowisku, w kt�rym zosta� zainstalowany. 
Przyk�adowo, kiedy pojawiaj� si� nowe dane w zewn�trznej bazie danych, nale�y 
przetestowa� aktualnie zgromadzon� wiedz� na nowych danych lub dokona� nowych 
oblicze�. Wymaganie to realizowa� b�d� tzw. \textbf{agenty} (link do opisu w dok.).
Poniewa� system b�dzie m�g� pracowa� w �rodowisku rozproszonym, 
ka�dy z agent�w pracuj�cych na r�nych maszynach mo�e niezale�nie wygenerowa� now� wiedze, 
po czym wiedza ta powinna zosta� rozdystrybuowana do pozosta�ych agent�w.
Takie podej�cie pozwala na bezobs�ugow� prace systemu. System sam powinien 
podejmowa� akacje w wyniku pewnych zdarze� w jego �rodowisku. 
Idea agent�w zosta�a zaczerpni�ta z Vinlena, kt�ry posiada 
wsparcie analogicznych mechanizm�w, zwanych tam skautami.
Stanowi�cych one integraln� cz�� tego systemu.
Weka nie posiada wsparcia dla takich mechanizm�w.
Jest ona bardziej narz�dziem do przeprowadzania akademickich 
eksperyment�w na algorytmach ni� narz�dziem do zastosowa� biznesowych, 
zainstalowane w �rodowisku produkcyjnym. System powinien posiada� cechy, 
kt�re umo�liwi� jego zastosowanie zar�wno w rozwi�zaniach biznesowych jak i 
akademickich.

% - rownoleglosc i rozproszenie
System, kt�ry mo�e wykonywa� skomplikowane i d�ugotrwa�e obliczenia powinien
mie� mo�liwo�� ich \textbf{rozpraszania}.
W dobie rozpowszechniania si� technologi gridowych i klastrowych aplikacje powinny
m�c korzysta� z mo�liwo�ci, jakie te technologie udost�pniaj�.
Dzia�anie w �rodowisku heterogenicznym pozwala nie tylko na lepsze 
wykorzystanie zasob�w sprz�towych, ale r�wnie� na dost�p do wielu �r�de� danych 
przy zapewnieniu bezpiecze�stwach tych danych. Rozproszone instancje powinny 
wymienia� jak najmniejsz� ilo�� informacji, by nie obci��a� namiernie zasob�w komunikacyjnych.

Istotna wydaje si� tak�e potrzeba zepewnienia mo�liwo�ci \textbf{zr�wnoleglenia} niekt�rych oblicze�.
Taka mo�liwo�� powinna by� dost�pna ramach pojedynczej instancji systemu,
co powinno by� uwzgl�dnione w jego architekturze.
W naszym przypadku w�a�ciwym i najprostszym modelem jest napisanie 
aplikacji pracuj�cych w oparciu o w�tki i przeniesienie ci�aru zrownolegniania 
na inne mechanizmy, takie jak klasteryzcja itp. Obecnie na rynku znajduj� si� systemy
pozowalaj�ce na zr�wnoleglenie aplikacji wielow�tkowej w spos�b prze�roczysty 
dla u�ytkownika jak i dla aplikacji. �wietnym przyk�adem takiej aplikacji jest 
np. MOSIX (http://www.mosix.org/), kt�ry z klastr�w linuxowych x86 tworzy 
system wieloprocesorowy.

W celu zapewnianiu \textbf{bezpiecze�stwa danych}, aplikacja nie powinna wysy�a� samych danych,
ale wy��cznie wiedz�. Nale�y r�wnie� za�o�y� ograniczone zaufanie do instancji 
pracuj�cych w nieznanym �rodowisku, dlatego ka�de �r�d�o danych powinno by� zabezpieczone.
W zastosowaniach komercyjnych system mo�e mie� dost�p do kluczowych danych 
przedsi�biorstwa czy instytucji. Model bezpiecze�stwa komunikacji mi�dzy 
r�nymi instancjami, chcieliby�my oprze� na obecnie dost�pnych metodach ochrony 
jak np. VPN (Virtual Private Network). Zak�adamy, �e system b�dzie zainstalowany w 
bezpiecznym �rodowisku, dlatego ochron� warstwy komunikacji powinien 
zaj�� si� administrator.

% TODO: to gdzie? (do wykorzystania algorytmow?)
%Wa�nym aspektem jest dost�p do algorytm�w Weki, 
%ale tak�e do ewentualnego u�ycia algorytm�w z VINLENA, co jedna wymaga 
%zastosowania odpowiedniej licencji (link do odpowiedniego punktu. VINLEN zosta� 
%napisany w j�zyku C++, zatem system powinnen wspiera� r�wnie� inne j�zyki 
%programowania a szczeg�lnie wspomniany C++.

% TODO - to gdzie?
%Kolejnym aspektem architektury jest spos�b aktualizacji 
%algorytm�w. Je�li w przypadku samej platformy aktualizacja nie b�dzie 
%dokonywana cz�sto to jednak w przypadku algorytm�w mo�e to nast�powa� bardzo 
%cz�sto. Przyk�adowym zastosowaniem wymagaj�cym cz�stych aktualizacji jest 
%wynajmowanie mocy obliczeniowej innym instytucjom. Mo�na wyobrazi� sobie 
%sytuacje, w kt�rej nasz system zainstalowany jest na wydajnej maszynie, np. klastrze.
%Wykonuje on obliczenia na zlecenie, co w oczywisty spos�b 
%wymaga�oby pobrania nie tylko parametr�w, czy danych, ale r�wnie� samych 
%algorytm�w. Oba referencyjne systemy nie spe�niaj� wymaganych przez nas 
%kryteri�w.

% -- Licencja
Istotn� cech� wp�ywaj�c� na popularno�� aplikacji jest spos�b ich \textbf{licencjonowania},
dlatego system powininen by� dost�pny na licencji, kt�ra umo�liwia� b�dzie jego
powszechne u�ywanie.
Vinlen to system komercyjny, dostarczany bez kodu �r�d�owego, 
dlatego jego wykorzystanie jest utrudnione zar�wno jako platformy dla w�asnych 
algorytm�w jak i jego algorytm�w. Z racji braku kodu �r�d�owego nie s� mo�liwe 
ewentualne modyfikacje i dostosowania do naszych potrzeb.
Weka dostarczana jest na licencji \emph{GPL}\footnote{GNU General Public License 
-- jedna z licencji wolnego oprogramowania, pozwala na korzystanie z programu w dowolnym celu, 
badanie, w jaki spos�b program dzia�a i modyfikowanie go, dystrybuowanie kopii, 
ulepszania programu i udost�pniania poprawek innym. 
(\href{http://www.gnu.org/licenses/gpl.html}{http://www.gnu.org/licenses/gpl.html})}, 
kt�ra jest licencj� open source, jednak r�wnie� stwarza pewne problemy. 
W przypadku tej licencji niemo�liwe jest u�ycie aplikacji w innym systemie o 
zamkni�tym kodzie �r�d�owym.
Licencja, na kt�rej b�dzie dost�pny system powinna umo�liwia� wykorzystanie 
go w zamkni�tych, komercyjnych projektach, a zarazem po��dan� cech� jest to, 
aby by�a zgodna z GPL. Zgodno�� licencji jest wymagana do ewentualnego u�ycia 
algorytm�w Weki w systemie. Po wst�pnej analizie dost�pnych licencji, 
jedn� z tych, kt�ra spe�nia nasze, wymagania jest licencja 
\emph{LGPL}\footnote{GNU Lesser General Public License 
-- licencja wolnego oprogramowania, stanowi�ca kompromis pomi�dzy GNU GPL a licencjami liberalnymi.
Nak�ada ograniczenia okre�lane jako \emph{copyleft} na poszczeg�lne pliki �r�d�owe,
ale nie na ca�y program, pod warunkiem u�ywania odpowiedniego mechanizmu bibliotek wsp�dzielonych 
(ang. shared library) oraz przestrzegania pewnych dodatkowych ogranicze�
(\href{http://www.gnu.org/licenses/lgpl.html}{http://www.gnu.org/licenses/lgpl.html})}.

Podsumowanie powy�szych wymaga� prowadzi do stwierdzenia,
�e system powinien cechowa� si� architektur� komponentow�, 
a jego funkcjonalno�� powinna by� �atwo rozszerzalna, dzi�ki dobrze zdefiniowanemu \emph{API}. 
Rozszerzalno�� ta b�dzie zapewniona poprzez zewn�trzne komponenty pochodz�ce od r�nych dostawc�w, tzw. \emph{wtyczki}.
Dzi�ki temu mo�liwe b�dzie wykorzystanie gotowych i sprawdzonych algorytm�w.
Jego architektura powinna wspiera� wykorzystanie mo�liwo�ci maszyn wieloprocesorowych  
oraz klastr�w st�d konieczno�� zapewnienia prowadzenia oblicze� w �rodowisku rozproszonym.
To poci�ga za sob� u�ycie j�zyka programowania wspieraj�cego tworzenie wielow�tkowych aplikacji
rozproszonych np. j�zyka \emph{Java}.
Jedn� z cech wp�ywaj�cych na jego u�yteczno�� powinien by� tak�e przyjazny
i intuicyjny graficzny interfejs u�ytkownika oraz udost�pnienie go na licencji
pozwalaj�cej na jak najszersze jego u�ytkowanie, jak� jest licencja LGPL.

%II. Salomon
%1. Napisa� postulaty - dlaczego piszemy Salomona:
%Funkcje:
%rozproszenie
%- wykorzystanie gotowych algorytm�w
%bezpiecze�stwo
%- komponentowa architektura
%- podej�cie do data set i attribute set
%- jedna reprezentacja wiedzy dzielona pomi�dzy ro�ne algorytmy
%- przeno�no��
%- dost�p nie tylko do wiedzy, ale tak�e do danych (indukcyjna baza danych)
%Api i licencja:
%- proste API
%- licencja
%

%TODO
%1. Rozproszenie
%2. Wykorzystanie gotowych algorytmow (vinlen wszystko sam pisze)
%3. Komponentowa architektura
%4. Proste API
%5. Otwarta licencja
%6. Podejscie do datasetow i atrybutow
%7. Jedna reprezentacja wiedzy (w wece drzewa nie maja wpolnej reprezentacji przez co niezaleznie rozwijane algorytmy moga ze soba wspolpracowac

%2. Uzasadnienie potrzeby stworzenia platformy
%	
%- dwa aspekty tematyki uczenia maszynowego:
%	* indukcyjne bazy danych (i Vinlen)
%	* eksperymenty i doswiadczenia (Weka itd.)
%- z tego ma wynikac koncepcja Salomona:
%	* jako platformy laczacej te aspekty
%	* odpowiedz na ograniczenia np. Vinlena
%- przedstawic postulaty, ktore spelaniac ma platfroma
%	* zalozenia koncepcyjne (np. koncepcja zadan)
%	* zalozenia implementacyjne (przenosnosc, rownoleglosc itp).
\section{Koncepcja systemu}

\subsection{Og�lne za�o�enia}

Bardzo wa�nym mechanizmem Salomona jest mo�liwo�� tworzenia powi�za� mi�dzy poszczeg�lnymi zadaniami. Wynik jednego zadania mo�e pos�u�y� jako wej�cie nast�pnego. Salomon w tym celu dostarcza �rodowisko. Wtyczka w trakcie wykonywania zadania opr�cz dost�pu  do manad�er�w posiada r�wnie� dost�p do �rodowiska, z kt�rego mo�e pobra� warto�ci zmiennych �rodowiskowych. Dane �rodowisko tworzone jest na pocz�tku wykonania listy zada� i przekazywane jest kolejno do nast�pnego zadania. Wtyczka w trakcie pracy mo�e modyfikowa� �rodowisko poprzez dodawanie, usuwanie oraz edycj� poszczeg�lnych zmiennych. W ten spos�b poszczeg�lne zadania mog� mi�dzy sob� przekazywa� informacje. Zmienne �rodowiskowe, podobnie jak ustawienia i rezultaty zada�, s� persystentne, a co za tym idzie potrzebny jest mechanizm serializacji i deserializacji.

W obecnej wersji mechanizm ten jest bardzo ubogi -- pozwala on na przekazywanie wy��cznie danych tekstowych. Wraz z pojawieniem si� kolejnej wersji, zmienne �rodowiskowe wykorzystywa� b�d� nowy mechanizm serializacji. Mo�liwo�� tworzenia zmiennych �rodowiskowych przez wtyczki mo�e rodzi� wiele problem�w z zapewnieniem kompatybilno�ci mi�dzy r�nymi wersjami wtyczek, komunikuj�cych si� ze sob�, dlatego te� aby unikn�� takich problem�w, ka�da wtyczka zmuszona b�dzie dostarczy� opis zmiennych, kt�re mo�e tworzy�.

W kolejnej wersji Salomona dodane zostan� klasy odpowiedzialne za przechowywanie danych. Klasy te b�d� odpowiada� poszczeg�lnym typom prostym oraz strukturze (struktura b�dzie mog�a zawiera� inne struktury oraz typy proste). Za pomoc� takich element�w, programista b�dzie m�g� stworzy� hierarchiczne struktury danych. Wprowadzenie takiego mechanizmu podyktowane jest potrzeb� ukrycia przed programist� sposobu zapisu danych w bazie lub w pliku. W obecnej wersji Salomona, tw�rca wtyczki musi dostarczy� mechanizm serializacji oraz deserializacji ustawie� i rezultat�w zada� do napisu -- taki mechanizm niesie za sob� niebezpiecze�stwo niepoprawno�ci oraz nieefektywno�ci implementacji. Dostarczenie sp�jnego modelu serializacji danych pozwala na niewidoczny dla wtyczek spos�b podmiany implementacji na bardziej efektywn� itp. Mechanizm ten mo�e okaza� si� r�wnie� po�yteczny po dodaniu do Salomona mo�liwo�ci definiowania zada� w pliku (np. XML). Serializacja danych mo�e odbywa� si� do wielu format�w np. XML, CSV, tabel w bazie danych itp.

\subsection{Podstawowe poj�cia}

\paragraph{Kontroler}
Kontroler to jedna instancja Salomona pracuj�ca w okre�lonym trybie: lokalnym, jako serwer lub klient. 

\paragraph{Menad�er} 
Klasy maj�ce w nazwie s�owo \emph{Manager} pe�ni� szczeg�ln� funkcj� - najcz�ciej odpowiadaj� za zarz�dzanie innymi klasami, z regu�y tymi, kt�re stanowi� reszt� nazwy (np. \emph{PluginManger} zarz�dza pluginami, \emph{ProjectManger} - projektami). G��wnym zadaniem menad�er�w jest odseparowanie wtyczek od operacji bezpo�rednio na danych.

\paragraph{Solution} 
Terminem tym okre�lamy struktur� grupuj�c� projekty. Dotycz� one tej samej bazy z rzeczywistymi danymi.

\paragraph{Projekt}
W ramach projekt�w zapisywana jest konfiguracja systemu przed wykonaniem kolejki zada�. Zapisywana jest lista zada� do wykonania, ich ustawienia oraz pluginy, kt�re maj� by� u�yte do ich wykonania.

\paragraph{Plugin (wtyczka)}
Jest to zewn�trzny program odpowiedzialny za wykonanie konkretnego zadania. Nie jest integraln� cz�ci� systemu, w razie konieczno�ci jest pobierany z okre�lonej lokalizacji i przekazywane jest mu konkretne zadanie do wykonania.

\paragraph{�rodowisko}
Miejsce w kt�rym przechowywane s� zmienne �rodowiskowe.

\paragraph{Zmienna �rodowiskowa}
S�u�y do przekazywania informacji mi�dzy kolejnymi zadaniami.

\section{Proponowana struktura systemu}

Proponowana struktura systemu zak�ada jego podzia� na dwie zasadnicze
cz�ci (Rys. \ref{fig:Achitecture}). Pierwsz� z~nich
stanowi platforma. Tworz� j�:
\begin{itemize}
	\item silnik (\emph{Salomon Core}), kt�rego zadaniem jest zarz�dzanie i~uruchamianie
zada�
	\item magazyn (\emph{Internal Database}) s�u��cy do przechowywania wiedzy
\end{itemize}

Druga cz�� to~zbi�r wtyczek, dostarczaj�cych logik� potrzebn� do skonfigurowania, wykonania i~wy�wietlenia rezultat�w
zadania. System jest otwarty - oznacza to, �e jego funkcjonalno�� mo�e by� w~�atwy spos�b rozszerzana poprzez dostarczenie nowych wtyczek.

\begin{figure}[H]
	\centering
		\includegraphics[width=0.90\textwidth]{img/salomon/concept/architecture.png}
	\caption{Architektura systemu}
	\label{fig:Achitecture}
\end{figure}


\pagebreak
Jednym z podstawowych za�o�e� projektowych jest umieszczenie algorytm�w
\emph{machine learning} we~wtyczkach. J�dro systemu ma by� jak najmniejsze --
Jego zadaniem jest stworzenie �rodowiska do realizacji logiki dostarczanej we wtyczkach. \\
Z tego za�o�enia wynika konieczno�� zastosowania otwartej, �atwo
rozszerzalnej architektury, umo�liwiaj�cej wprowadzenie warstwy po�redniej pomi�dzy baz� danych~a wtyczkami. Zadaniem tej warstwy jest ukrycie sposobu organizacji danych przed wtyczkami.\\
W celu zapewnienia przeno�no�ci i �atwego rozpraszania oblicze� system powinien
by� niezale�ny od platformy, co umo�liwi uruchamianie jego poszczeg�lnych cz�ci
w �rodowisku heterogenicznym. \\
Dla uzyskania odpowiedniego poziomu skalowalno�ci wymagana jest dostatecznie du�a granulacja procesu przetwarzania wiedzy, a zatem konieczne jest
dostarczenie mechanizmu komunikacji pomi�dzy poszczeg�lnymi etapami tego procesu.

\section{Funkcjonalno��}

Ca�a funkcjonalno�� \emph{Salomona} podporz�dkowana jest tworzeniu zada�. Operacje podstawowe, takie jak np. tworzenie projekt�w czy definiowanie wtyczek s�u�� jedynie stworzeniu �rodowiska, w kt�rym mog� by� wykonywane zadania. Z kolei operacje administracyjne pozwalaj� na zaawansowane zarz�dzanie platform� i mog� by� u�yteczne dla u�ytkownik�w bardzo dobrze znaj�cych wewn�trzn� struktur� systemu.

\subsection{Operacje podstawowe}

\subsubsection{Operacje na obiekcie Solution}

Terminem tym okre�lamy struktur� grupuj�c� projekty.
W obr�bie jednego obiektu \emph{Solution} zgrupowane s� projekty, 
kt�re operuj� na tym samym zewn�trznym �r�dle danych.
W obecnej wersji s� to zewn�rzne bazy danych. Obiekt ten przechowuje 
informacje niezb�dne do uzyskania dost�pu do danych,
a wi�c parametry po��czenia i lokalizacj� bazy danych.

Mo�na na nim wykona� nast�puj�ce operacje:

\begin{itemize}
	\item Utworzenie nowego
		
	Zadanie to wymaga konfiguracji, w szczeg�lno�ci wskazania lokacji bazy danych i innych parametr�w po��czenia.
		
	\item Edycja istniej�cego
	
	System pozwala na wyb�r aktualnego \emph{Solutiona}. Mo�na to zrobi� przy uruchamianiu programu, kiedy to trzeba albo wskaza� istniej�cy \emph{Solution} albo utworzy� nowy, b�d� w dowolnym innym momencie, poprzez wyb�r odpowiedniej opcji z menu. Tylko w ten spos�b mo�na zmieni� parametry ju� istniej�cego obiektu \emph{Solution}.	

\end{itemize}

\subsubsection{Operacje na projektach}

Projekty s�u�� do grupowania zada� w obr�bie tego samego obiektu \emph{Solution}.			
W ramach projekt�w zapisywana jest konfiguracja systemu przed wykonaniem grupy zada�.
Zadania te grupowane s� w formie grafu skierowanego acyklicznego.
Struktura taka opisuje zestaw zada�, kt�re musz� zosta� wykonane w okre�lonej kolejno�ci
aby uzyska� zamierzony efekt. Przyk�adowo, aby stworzy� i wy�wietli� drzewo decyzyjne,
mo�na zdefiniowa� zadania dla poszczeg�lnych krok�w, a wi�c wybrania zbioru danych, zdefiniowania zbioru atrybut�w,
skonfigurowania algorytmu tworz�cego drzewo, czy wreszcie jego wizualizacji.
Projekty mog� by� u�yte wielokrotnie dla r�nych danych i r�nej konfiguracji zada�. 

Operacje na projekcie:		
	\begin{itemize}	
		\item Utworzenie projektu
		
		Nowy projekt tworzony jest automatycznie podczas uruchomienia systemu. Mo�na te� wymusi� utworzenie nowego projektu poprzez wyb�r odpowiedniej opcji z menu.
		
		\item Edycja bie��cego projektu
		
		Operacja ta pozwala na zmian� podstawowych informacji o projekcie, a wi�c jego nazwy i dodatkowych informacji.
			
		\item Uruchomienie projektu
		
        Akcja ta powoduje wykonanie zada� zgromadzonych w tym projekcie. Wykonanie tej akcji wymaga wcze�niejszej konfiguracji w postaci specyfikacji listy zada� do wykonania oraz ich indywidualnej konfiguracji.
        
        
	\end{itemize}
	
\subsubsection{Operacje na wtyczkach}

Wtyczka to zewn�trzny program odpowiedzialny za wykonanie konkretnego zadania.
Mo�e zawiera� algorytm obliczeniowy, ale nie tylko -- wtyczki mog� te� s�u�y�
np. do definiowania nowych zbior�w danych czy do wy�wietlania drzew decyzyjnych
w formie graficznej. Ka�da wtyczka mo�e by� skonfigurowana przed wykonaniem jej 
g��wnego zadania. Za obs�ug� konfiguracji odpowiada platforma, tw�rca wtyczki musi
jedynie dostarczy� komponent�w pozwalaj�cych na skonfigurowanie ustawie� wtyczki
z poziomu graficznego interfejsu u�ytkownika.
Wtyczki nie s� integraln� cz�ci� systemu. W razie konieczno�ci mog� by� pobierane 
z okre�lonej zdalnej lokalizacji.
Wtyczki zawieraj� algorytmy, kt�re mog� by� wykonywane przez platform� w ramach poszczeg�lnych zada�.
S� one podstawowymi komponentami, kt�re rozszerzaj� mo�liwo�ci systemu.

	\begin{itemize}
	
		\item Dodanie
		
			Operacja ta pozwala na zdefiniowanie nowej wtyczki poprzez wskazanie odpowiedniego pliku j� reprezentuj�cego.
		
		\item Edycja
		
			Za pomoc� tej operacji mo�na zmienia� nazw� i opis wcze�niej zdefiniowanej wtyczki.
		
		\item Usuni�cie
		
			Pozwala na usuni�cie wtyczki z systemu.
					
	\end{itemize}


\subsection{Operacje na zadaniach}

Definiowanie zada� do wykonania nale�y do podstawowych funkcji systemu, dlatego w obecnej wersji ich definiowanie i konfigurowanie zosta�o znacznie rozbudowane.

Zadanie to podstawowa jednostka obliczeniowa.
Zawiera algorytm, kt�ry ma zosta� wykonany oraz jego ustawienia.
Algorytm dostarczany jest w formie wtyczki. Szerszy opis zada� znajduje si� w rozdziale \ref{lab:tasks}.

Pierwsze wersje \emph{Salomona} pozwala�y jedynie na definiowanie liniowej listy zada�.
Dlatego, aby umo�liwi� tworzenie bardziej zaawansowanych struktur zada� zosta� stworzony nowy komponent oparty o bibliotek� \emph{Jung}.
Zadania reprezentowane w nim s� jako w�z�y w grafie skierowanym acyklicznym. Kraw�dzie reprezentuj� przep�yw sterowania (wiedzy/danych).

Aby dane zadanie mog�o by� wykonane wszystkie zadania od kt�rych zale�y musz� by� wykonane wcze�niej.
W obecnej wersji acykliczno�� grafu jest wymagana, aby zapewni�, �e wykonanie programu si� zako�czy.
W kolejnych wersjach planowane jest umo�liwienie dodawania warunk�w na kraw�dzie, a tym samym
mo�liwo�ci tworzenia cykli, kt�re zostan� przerwane, je�li zostanie spe�niony odpowiedni warunek.

Obecnie obs�ugiwane s� nast�puj�ce operacje na zadanich:

		\begin{itemize}
			\item Dodanie
			
			Dodanie zadania polega na zdefiniowaniu jego nazwy i wskazaniu wtyczki, kt�ra zawiera odpowiedni algorytm.
			
			\item Usuni�cie
			
            Akcja usuwa aktualnie zaznaczone zadnie.             
            
            \item Okre�lanie kolejno�ci wykonania
            
            Zadania wykonywane s� tak, jak zosta�y po��czone kraw�dziami. Strza�ki na kraw�dziach okre�laj� kolejne zadania do wykonania. W�z�y grafu, kt�re reprezentuj� poszczeg�lne zadania mog� by� w dowolny spos�b przemieszczane oraz ��czone.
            
        \item Definiowanie ustawie� zadania
        
        Ka�de zadanie mo�e by� skonfigurowane przed wykonaniem. Spos�b konfiguracji zale�y od poszczeg�lnych wtyczek i to one dostarczaj� komponenty pozwalaj�ce na skonfigurowanie danej wtyczki.
        
        \item Uruchomienie zada�
        
        Po zdefiniowaniu kolejno�ci wykonania zada� mo�na je uruchomi� -- zadania zostan� przetworzone, a ich wyniki zapisane w bazie danych.
        
  		\item Przegl�danie rezultat�w
        
        Ka�de wtyczka, kt�rej algorytm przetwarzany jest w ramach zadania, sama specyfikuje jak b�dzie prezentowa� swoje wyniki. Zdaniem platformy jest jedynie zapewnienie mo�liwo�ci wy�wietlenia zwr�conych przez wtyczk� danych.
		\end{itemize}

\subsection{Administracja}

\emph{Salomon} dostarcza narz�dzi u�atwiaj�ce administrowanie danymi zgromadzonymi w systemie -- mo�liwe jest przegl�danie wszystkich zapisanych w bazie danych obiekt�w, a zaawansowani u�ytkownicy mog� wykona� dowolne operacje bezpo�rednio na bazie danych za pomoc� \emph{SQLConsole}.
	
	\begin{itemize}
            
		\item Przegl�danie obiekt�w w systemie
		
		W celach administracyjnych mo�liwe jest przegl�danie obiekt�w w taki spos�b, w jaki zapisane s� w bazie danych oraz na ich usuni�cie.
			
		\item Zaawansowane operacje
		
		Do bardziej zaawansowanych operacji na bazie danych przeznaczona jest specjalna konsola (\emph{SQLConsole}), kt�ra pozwala na wykonanie dowolnych operacji na bazie danych.
		
	\end{itemize}	

%%
%% ==== ROZDZIA� 3 ====
%%
\chapter{Wybrane aspekty implementacji}

\section{Struktura systemu}

System zosta� podzielony na 3 g��wne cz�ci: platform�, kontrolery
i~pluginy.

\subsection{Platforma}

Dostarcza podstawowej funkcjonalno�ci  umo�liwiaj�cej prac� ca�ego
systemu,  �aduje odpowiedni kontroler, wczytuje wtyczki, uruchamia
zadania. Odpowiada za komunikacje mi�dzy innymi instancjami \emph{Salomona}.

\begin{figure}[H]
	\centering
		\includegraphics[width=1.00\textwidth]{img/uml/manager_engine.png}
	\caption{G��wne klasy platformy}
	\label{fig:core}
\end{figure}

Za poszczeg�lne zadania odpowiadaj� odpowiednie menad�ery.
Menad�ery nie s� dost�pne bezpo�rednio z platformy, ale
przekazywane s� wtyczk� poprzez klas� \emph{IDataEngine}. Dotyczy
to tylko klasy \emph{IDataSetManager} i \emph{IRuleSetManager},
pozosta�e nie s� dost�pne dla wtyczek. Plugin, zale�nie od
potrzeb, pobiera z niego potrzebny mu menad�er i za jego
po�rednictwem wykonuje operacje na bazie danych.

\paragraph{IDataEngine}

Obiekt implementuj�cy ten interfejs przekazywany jest wtyczkom podczas ich wykonania.
Za pomoc� pobieranych z niego menad�er�w (\emph{IDataSetManager}, \emph{IAttributeManager}, w przysz�o�ci \emph{IRuleSetManager} i inne) wtyczki mog� operowa� na danych znajduj�cych si� w bazie.

\paragraph{DBManager}

Odpowiada za po��czenie z baz� danych. Dostarcza metody
zapewniaj�ce dost�p do danych przechowywanych w bazie. Swoj�
funkcjonalno�� udost�pnia odpowiednim menad�erom.

\paragraph{IDataSetManager}

Zarz�dza zbiorami danych. Pozwala tworzy� nowe podzbiory danych na
podstawie zawartych w bazie informacji oraz umo�liwia operowanie
na nich.

\paragraph{IRuleSetManager}

Zarz�dza regu�ami. Pozwala tworzy� nowe regu�y oraz
zarz�dza dost�pem do ju� istniej�cych.

\paragraph{IAttributeManager}

Zarz�dza atrybutami. Pozwala tworzy� nowe atrybuty oraz
zarz�dza dost�pem do ju� istniej�cych.


\paragraph{IManagerEngine}

Zarz�dza pozosta�ymi menad�erami. Utrzymuje jedn� instancj� ka�dego z nich 
i udost�pnia je pozosta�ym klasom z platformy.

\paragraph{ISolutionManager}

Zarz�dza obiektami implementuj�cymi interfejs \emph{ISolution}. 
Dostarcza metod pozwalaj�cych na utworzenie nowego obiektu \emph{ISolution},
modyfikacj� aktualnego, za�adowanie wcze�niej stworzonego lub zwr�cenie wszystkich obiekt�w \emph{ISolution}.

\paragraph{IProjectManager}

Zarz�dza projektami. Pozwala na utworzenie nowego
projektu, zapisanie bie��cego do bazy danych oraz na za�adowanie z bazy.

\paragraph{IPluginManager}

Zarz�dza pluginami. Pozwala na zapisanie informacji o nowym pluginie (jego nazwa, lokalizacja) oraz na pobranie listy dost�pnych plugin�w.

\paragraph{ITaskManager}

Zarz�dza zadaniami. Jego zadaniem jest nie tylko umo�liwienie utworzenia lub edycji istniej�cych zada�, ale tak�e nadzorowanie ich uruchamiania.

\subsection{Kontrolery}
Kontrolery odpowiadaj� za interakcj�
systemu z otoczeniem. W zale�no�ci od konfiguracji systemu przy
starcie uruchamiany jest jeden z kontroler�w. Kontrolery operuj�
na danych poprzez wsp�lny interfejs, a co za tym idzie, dane
utworzone poprzez jeden z nich s� dost�pne pomi�dzy kolejnymi
uruchomieniami programu dla pozosta�ych kontroler�w.

\paragraph{LocalController}

Jest najprostszym kontrolerem. Zarz�dza zadaniami wykonywanymi na lokalnym komputerze. S� one
wykonywane sekwencyjnie. Zadaniem tego kontrolera jest dostarczenie
interfejsu u�ytkownika, pozwalaj�cego na zarz�dzanie projektami,
wtyczkami i zadaniami.

\paragraph{MasterController}

Zadaniem tego kontrolera jest dostarczenie interfejsu do
zarz�dzania zdalnymi kontrolerami (\emph{ServantController}). 
Po dodaniu warstwy \emph{Solution} kontroler ten musi zosta� zmodyfikowany, co planowane jest w nast�pnej wersji systemu.
%Po uruchomieniu nas�uchuje na po��czenia od klient�w, rozdziela
%zadania oraz wy�wietla ich wyniki.

\paragraph{ServantController}

Zadaniem tego kontrolera jest odszukanie g��wnego kontrolera
(\emph{MasterController}), zarejestrowanie si� i udost�pnianie mu
swoich us�ug. Ta wersja kontrolera nie posiada GUI, zarz�dzanie
nim odbywa si� za pomoc� klasy \emph{MasterController}.
Podobnie jak w przypadku klasy \emph{MasterController}, kontroler ten musi zosta� zmodyfikowany.

%\paragraph{WinSalomon}

%\section{WinSalomon}

\emph{WinSalomon} to projekt maj�cy na celu stworzenie biblioteki DLL, za pomoc� kt�rej mo�liwe b�dzie uruchamianie i kontrolowanie Salomona.

Protoplasta \emph{Salomona} - \emph{Vinlen} - zosta� napisany w \emph{C++} (\emph{C++ Builder}). Graficzny interfejs tego �rodowiska jest bardzo zaawansowany i dostarcza znacznie wi�kszych mo�liwo�ci ni� \emph{Salomon}. Aby mo�liwe by�o wykorzystanie tylko logiki \emph{Salomona} i pozostawienie graficznego interfejsu \emph{Vinlena} konieczne by�o stworzenie warstwy ��cz�cej  GUI napisane w \emph{C++} z logik� napisan� w \emph{Javie}.

Mo�liwe sposoby po��czenie \emph{Vinlena} z \emph{Salomonem}:
\begin{itemize}
	\item \emph{JNI} - Java Native Interface
	\item CORBA
	\item inne sposoby komunikacji mi�dzy procesowej (pami�� dzielona, pipe'y, komunikaty)
\end{itemize}

W celu realizacji projektu wybrali�my \emph{JNI}. Przemawia�o za nim:
\begin{itemize}
	\item wydajno�� implementacji
	\item prostota
\end{itemize}

Etapy projektu
\begin{enumerate}
	\item LibraryController

Pierwszym etapem projektu by�o stworzenie kontrolera, kt�ry umo�liwia�by wykorzystywanie \emph{Salomona} nie jako odr�bnej aplikacji, ale jako biblioteki dla program�w dzia�aj�cych w �rodowisku \emph{Java}.	

	\item WinSalomon DLL

Drugim etapem by�o stworzenie biblioteki DLL, w kt�rej znajdowa�yby si�	,,wrappery'' dla wszystkich interfejs�w udost�pnianych przez \emph{LibraryControllera}. Klasy te ukrywaj� przez u�ytkownikiem fakt, �e komunikacja odbywa si� przez \emph{JNI}, wi�c nic nie stoi na przeszkodzie, by dorobi� wersj� \emph{WinSalomona}, kt�rej wewn�trzna imlementacja opiera�aby si� np. na technologii \emph{CORBA}.
	
\end{enumerate}





\subsection{Wtyczki}
G��wna funkcjonalno�� zosta�a  przeniesiona
do wtyczek, zadaniem systemu jest tylko zarz�dzanie ich
wykonaniem. Dzi�ki takiemu podej�ciu system jest �atwo skalowalny
i rozszerzalny o nowe mo�liwo�ci. Ka�da z wtyczek musi
implementowa� nast�puj�ce interfejsy:

\paragraph{IGraphicPlugin}

Pozwala pobra� parametry od u�ytkownika, kt�re nast�pnie zostan�
przekazane do wtyczek  przed ich wykonaniem. Zawiera dwie metody:
\emph{getSettingsPanel()}  i \emph{getResultPanel()}.  Pierwsza z
metod zwraca panel s�u��cy do konfiguracji pluginu, druga -- panel,
na kt�rym prezentowane s� wyniki jego dzia�ania.

\paragraph{IDataPlugin}

Posiada tylko jedn� metod� \emph{doJob()}. Przyjmuje ona jako
parametry obiekt klasy \emph{Environment}, reprezentuj�cy aktualny
stan systemu; \emph{IDataEngine}, kt�ry umo�liwia operowanie na
bazie danych i \emph{ISettings}, reprezentuj�cy ustawienia
wtyczki. Zwracany jest obiekt klasy \emph{IResult}, stanowi�cy
rezultat wykonania zadania.

\section{Tworzenie w�asnej wtyczki}

System, dzi�ki swej elastycznej architekturze, pozwala u�ytkownikowi na 
rozszerzanie jego mo�liwo�ci poprzez definiowanie przez niego w�asnych 
wtyczek. Ca�a funkcjonalno�� zwi�zana z ich konfiguracj�, przetwarzaniem, prezentacj� wynik�w dzia�ania oraz komunikacj� przez 
sie� ukryta jest przed tw�rc� wtyczki i nie musi by� brana pod uwag� przy jego 
projektowaniu i implementacji. Jedyne, co musi on zrobi� to zaimplementowa� 
opisane poni�ej interfejsy. By system m�g� skorzysta� z nowych wtyczek nale�y 
poda� ich lokalizacje, umo�liwiaj�c systemowi tym samym pobranie i 
za�adowanie wtyczki. Obecna wersja systemu obs�uguje tylko wtyczki w postaci 
archiw�w jar.

%Poni�ej zamieszczony jest schemat przyk�adowego pluginu:

%\begin{figure}[!hb] \centering 
%\includegraphics[scale=0.40]{img/salomon/uml/average_price_plugin.jpg}
%	\caption{Przyk�adowy plugin}
%	\label{fig:AveragePricePlugin}
%\end{figure}

%Klasy zaznaczone na zielono i rozpoczynaj�ce si� od liter \emph{AP} to 
%przyk�adowe implementacje interfejs�w plugni�w (kolor ��ty).

\subsection{Interfejsy plugin�w}

Ka�da wtyczka aby mog�a by� przetwarzana przez system, musi implementowa� 
nast�puj�ce \mbox{interfejsy:}

\subsubsection{IPlugin}

\subsubsection{IDataPlugin}
	\begin{itemize}
		\item \emph{doJob()} - metoda wykonuje g��wne zadanie wtyczki. Jej parametrami s�:
		\begin{itemize}
          \item \emph{IDataEngin}, za pomoc� kt�rego wtyczka mo�e operowa� na 
          danych i wiedzy.
          \item \emph{IEnvironment} to interfejs umo�liwiaj�cy komunikacje 
          pomi�dzy poszczeg�lnymi zadaniami
          \item \emph{ISettings} konfiguracja wej�ciowa wtyczki
        \end{itemize}  
    \end{itemize}
    
\subsubsection{IGraphicPlugin}
\begin{itemize}
	\item \emph{getSettingComponent()} - zwraca obiekt implementuj�cy interfejs 
		\emph{ISettingComponent}
	\item \emph{getResultComponent()} - zwraca obiekt implementuj�cy interfejs 
		\emph{IResultComponent}\\
\end{itemize}

\subsubsection{ISettingComponent}
\begin{itemize}
	\item \emph{getComponent(ISettings settings)} - zwraca graficzny komponent 
s�u��cy do edycji ustawie� pluginu. Jest on wype�niany warto�ciami przekazanymi 
w obiekcie \emph{ISettings}.
	\item \emph{getSettings()} - zwraca ustawienia wtyczki.
	\item \emph{getDefaultSettings()} - zwraca domy�lne ustawienia wtyczki. 
Wykorzystywane, gdy u�ytkownik nie wprowadzi� w�asnych ustawie�.
\end{itemize}    

\subsubsection{IResultComponent}
\begin{itemize}
	\item \emph{getComponent(IResult result)} - zwraca graficzny komponent 
wy�wietlaj�cy rezultat wykonania wtyczki. Wype�niany jest on na podstawie 
obiektu  \emph{IResult} zwr�conego wcze�niej przez metod� \emph{doJob()}.
	\item  \emph{getDefaultResult()} - zwraca domy�lny rezultat wykonania wtyczki.
\end{itemize}

\subsubsection{IObject} To interfejs zapewniaj�cy persystencj� takich obiekt�w 
jak ustawienia, rezultat wykonania wtyczki, czy obiekty wykorzystywane do komunikacji mi�dzy wtyczkami.

\subsubsection{ISettings}Implementowany przez obiekt reprezentuj�cy ustawienia 
wtyczki.
\begin{itemize}
	\item \emph{init(IObject o)} - metoda inicjalizuje obiekt klasy
\end{itemize}    

\subsubsection{IResult} Implementowany przez obiekt reprezentuj�cy rezultat 
dzia�ania wtyczki.
\begin{itemize}
	\item \emph{init(IObject o)} - metoda inicjalizuje obiekt klasy
\end{itemize}  

%WSTAWI� TO W JAKIE� MIEJSCE!!!
%
%\paragraph{Persystencja danych}
%Z racji na rozproszony charakter aplikacji, oraz na wielko�� oblicze�.
%Wa�nym aspektem platformy jest zapewnienie uniwersalnego i wydajnego
%mechanizmu persyste�cji danych. W przypadku wewn�trznych struktur danych,
%jak zapis drzewa decyzyjnego, czy atrybut�w, zadaniem tym zajmuj� si�
%odpowiednie menadz�ry. W przypadku jednak persystencji danych u�ytkownika,
%takich jak ustawienia wtyczek, czy ich rezultaty zadaniem tym zajmuj� si�
%modu� persyste�cji, udost�pniaj�cy odpowiedni \emph{API}. Dodatkowo mechanizm
%ten jest wykorzystywany do persystencji danych wykorzystywanych w komunikacji mi�dzy kolejnymi zadaniami (WSTAWI� LINKA DO OPISU ENV.)
%
%Kod poni�ej tworzy struktur�, z jednym polem o nazwie ,,myField'' oraz przypisuje do niego warto�� ,,my settings''.
%
%\begin{lstlisting}
%SimpleStruct struct = new SimpleStruct();
%struct.setField("myField", "my settings");
%\end{lstlisting}
%
%WSTAWI� UML? 


%%
%% ==== ROZDZIA� 4 ====
%%
\chapter{Interfejs u�ytkownika}
\section{Interfejs u�ytkownika}

\subsection{Tryb serwera}
G��wnym zadaniem systemu jest umo�liwienie definiowania i wykonywania zada� w oparciu o dost�pne wtyczki. Gdy program dzia�a w trybie serwera po lewej stronie widoczna jest lista pod��czonych klient�w. Wszystkie opisane poni�ej operacje wykonywane s� dla aktualnie wybranego klienta.

\subsubsection{Zadania}
Aby zdefiniowa� nowe zadanie, nale�y wybra� wtyczk�, za pomoc� kt�rego b�dzie ono realizowane oraz skonfigurowa� jego ustawienia. Wtyczk� nale�y wybra� z listy dost�pnych, znajduj�cej si� na �rodku g��wnego okna programu. Po wybraniu wtyczki i naci�ni�ciu przycisku ze strza�k� w prawo znajduj�cego si� miedzy panelem z wtyczkami a panelem z zadaniami pojawia si� okienko, w kt�rym nale�y wprowadzi� nazw� zadania. 

\begin{figure}[htb]
	\centering
		\includegraphics[width=0.60\textwidth]{img/screenshot/new_task.jpg}
	\caption{Utworzenie nowego zadania}
	\label{fig:new_task}
\end{figure}


Po potwierdzeniu nazwy zostanie utworzone nowe zadanie i dodane do listy zada�. Ka�de z zada� mo�na skonfigurowa� przed wykonaniem, w tym celu nale�y nad wybranym zadaniem wywo�a� menu kontekstowe (prawym klawiszem myszy) i wybra� opcj� \textbf{Settings}. Po jej wybraniu pojawi si� okienko s�u��ce do konfiguracji zadania - mo�e by� r�ne, zale�nie od pluginu kt�ry ma by� wykonany w ramach zadania. 

\begin{figure}[htb]
	\centering
		\includegraphics[width=0.60\textwidth]{img/screenshot/task_settings.jpg}
	\caption{Ustawienia pluginu}
	\label{fig:task_settings}
\end{figure}


Po skonfigurowaniu zada� nale�y zatwierdzi� list� zada� do wykonania za pomoc� klawisza \textbf{Apply}. Je�li zadania nie zosta�y wcze�niej zapisane w ramach projektu, pojawi si� okienko w kt�rym nale�y poda� nazw� projektu. Po poprawnym zapisaniu projektu mo�na wykona� zaplanowane zadania. Zostan� one wykonane  w takiej kolejno�ci, w jakiej znajduj� si� one na li�cie zada�. Mo�liwa jest zmiana tej kolejno�ci - s�u�� do tego przyciski ze strza�kami znajduj�ce si� po prawej stronie panelu z zadaniami. Je�li zostanie ustalona ostateczna kolejno�� zada�, mo�na je wykona� za pomoc� przycisku \textbf{Run}.

\paragraph{}
Po wykonaniu zada� mo�na obejrze� ich rezultaty. W tym celu nale�y z menu kontekstowego dla zada� wybra� opcj� \textbf{Result}. Po jej wybraniu pojawi si� okienko prezentuj�ce wyniki wykonania danego zadania. Jego wygl�d, podobnie jak w przypadku ustawie� zada�, zale�y od pluginu.

\begin{figure}[htb]
	\centering
		\includegraphics[width=0.60\textwidth]{img/screenshot/task_result.jpg}
	\caption{Rezultat dzia�ania pluginu}
	\label{fig:task_result}
\end{figure}


\subsubsection{Pluginy}
Lista plugin�w tworzona jest na podstawie odpowiednich wpis�w w bazie danych. Mo�na j� jednak  modyfikowa� - dodawa�, usuwa� lub modyfikowa� dane pluginy. S�u�� do tego odpowiednie opcje z menu kontekstowego wywo�ywanego dla plugin�w. 

\begin{figure}[htb]
	\centering
		\includegraphics[width=0.60\textwidth]{img/screenshot/plugin_edit.jpg}
	\caption{Dodanie pluginu}
	\label{fig:plugin_edit}
\end{figure}


\subsubsection{Projekty}

Konfiguracja plugin�w oraz lista zada� zapisywana jest w ramach projekt�w. Dzi�ki temu mo�na zdefiniowa� r�ne listy zada� i wykonywa� je wielokrotnie, bez potrzeby ich ponownego tworzenia i konfiguracji. Aby za��dowa� istniej�cy projekt nale�y z menu g��wnego wybra� pozycj� \textbf{Projects$\rightarrow$Open Project}. Spowoduje to wy�wietlenie listy zapisanych projekt�w i umo�liwi wyb�r projektu do za�adowania. 

\begin{figure}[htb]
	\centering
		\includegraphics[width=0.60\textwidth]{img/screenshot/open_project.jpg}
	\caption{Wyb�r projektu}
	\label{fig:open_project}
\end{figure}


\subsection{Tryb klienta}
Program pracuj�cy w tym trybie nie posiada interfejsu u�ytkownika, jego konfiguracja odbywa si� z serwera.

\subsection{Tryb lokalny}
Interfejs dla programu dzia�aj�cego w trybie lokalnym zbli�ony jest do programu w trybie serwera. Nie posiada on jednak listy pod��czonych klient�w, gdy� sam wykonuje zdefiniowane zadania, czyli zachowuje si� jak po��czenie dw�ch program�w dzia�aj�cych w trybie serwera i klienta.

\subsection{SQL Console}
W trybie serwera oraz w trybie lokalnym z menu Tools dost�pny jest program SQL Console. Pozwala on na wykonywanie zapyta� SQL-owych, przez co umo�liwia administracj� projektami, pluginami i zadaniami oraz innymi tabelami wykorzystywanymi przez system.



%%
%% ==== ROZDZIA� 5 ====
%%

\chapter{Eksperyment}
\section{Opis eksperymentu}

Celem eksperymentu jest zademonstrowanie mo�liwo�ci platformy jako �rodowiska,
kt�re nadzoruje obliczenia uczenia maszynowego w oparciu o dostarczone wtyczki.

Przyk�adowe zadanie obliczeniowe ma na celu utworzenie drzewa decyzyjnego
w oparciu o zbi�r atrybut�w. Atrybuty te definiuj� w jaki spos�b interpretowany
ma by� zbi�r danych testowych. Zbi�r danych budowany jest w oparciu o dane
przechowywane w zewn�trznym �r�dle danych.

Model oblicze�, kt�ry wspiera plaftorma zak�ada du�� granulacj� zada�.
Powy�sze zadanie obliczeniowe mog�oby by� wykonane przez pojedyncz� wtyczk�,
kt�ra na wej�ciu wymaga�aby zdefiniowania zbioru danych i atrybut�w,
a na wyj�ciu zaprezentowa�aby wygenerowane drzewo.
Taka wtyczka by�aby jednak dosy� skomplikowana i jej wykorzystanie ograniczone
by�oby tylko do tego typu zada�.
Podzielenie ca�ego zadania na mniejsze etapy i wykonanie ka�dego z nich przez
osobn� wtyczk� pozwala nie tylko na wielokrotne wykorzystanie wtyczek do r�nych
cel�w, ale tak�e na wymian� wtyczek odpowiedzialnych za poszczeg�lne
etapy (np. zmian� algorytmu generuj�cego drzewo decyzyjne).

Podzia� eksperymentu na etapy zaprezentowany zosta� na rysunku (TODO: ref).

%TODO: zrobic taki rysunek

Poszczeg�lne etapy eksperymentu to:
\begin{enumerate}
	\item Zdefiniowanie zbioru danych. \\
	Celem tego etapu jest wygenerowanie zbioru danych w oparciu
	o podane przez u�ytkownika kryteria. Zbi�r danych wykorzystuje
	dane przechowywane w zewn�trznym �r�dle danych.
	\item Zdefiniowanie zbioru atrybut�w. \\
	Po wygenerowaniu zbioru danych tworzony jest zbi�r atrybut�w,
	na kt�rych dzia�a algorytm buduj�cy drzewo decyzyjne.
	\item Wygenerowanie drzewa decyzyjnego. \\
	Etap ten ma za zadanie wygenerowa� drzewo decyzyjne na podstawie
	utworzonego wcze�niej zbioru atrybut�w.
	\item Wizualizacja drzewa decyzjnego. \\
	Poprzedni etap tylko tworzy i zapisuje drzewo decyzyjne, 
	kt�re mo�e zosta� wykorzystane w dowolny spos�b.
	Ten etap eksperymentu jedynie prezentuje u�ytkownikowi wygenerowane
	wcze�niej drzewo w postaci graficznej.
\end{enumerate}

\section{Wykorzystane wtyczki}

\subsection{Definiowanie zbioru danych}

Testowy zbi�r danych definiowany jest za pomoc� wtyczki \emph{RandomDataSetCreator}.
Wtyczka ta pozwala na wygenerowanie losowego zbioru danych.
Konfiguracja wtyczki polega na wybraniu tabel z zewn�trznego �r�d�a danych,
na kt�rym bazowa� ma zbi�r danych oraz okre�lenia, ile wierszy z poszczeg�lnych
tabel ma zosta� do��czone do testowego zbioru danych (Rys \ref{fig:random_data_set_settings}).

\begin{figure}[!hb]
	\centering
		\includegraphics[scale=0.60]{img/experiment/random_dataset_settings.png}
	\caption{Konfiguracja losowego zbioru danych}
	\label{fig:random_data_set_settings}
\end{figure}

Prostota u�ycia tej wtyczki okupiona zosta�a niewielkimi jej mo�liwo�ciami.
Du�o wi�ksze mo�liwo�ci konfiguracyjne zapewnia wtyczka \emph{DataSetCreator}.
Posiada ona podobny graficzny interfejs u�ytkownika, ale pozwala na du�o dok�adniejsze
okre�lenie warunk�w, jakie spe�nia� maj� dane, kt�re trafi� do testowego zbioru danych (Rys \ref{fig:data_set_settings}).

\begin{figure}[!hb]
	\centering
		\includegraphics[scale=0.50]{img/experiment/dataset_settings.png}
	\caption{Okre�lanie dok�adnych kryteri�w przy tworzeniu zbioru danych}
	\label{fig:data_set_settings}
\end{figure}

Dodatkow� jej zalet� jest to, �e wtyczka ta w znacznie wi�kszym stopniu ni� \emph{RandomDataSetCreator}
uwzgl�dnia zmiany w �r�dle danych. Pierwsza wtyczka reaguje tylko na zmian� poszczeg�lnych
kolumn w wylosowanych do zbioru danych rekordach, nie mo�e si� jednak zmieni� ich ilo��.
Wtyczka \emph{DataSetCreator} uruchamiana wielokrotnie z t� sam� konfiguracj�
mo�e pracowa� na r�nej ilo�ci danych, zale�nie od tego, ile rekord�w spe�nia
zadane kryteria w momencie jej uruchomienia.
Pozwala to na przeprowadzanie ekperyment�w na zmieniaj�cych si� danych i przyk�adowo
na obserwacj�, jak ilo�� danych wp�ywa na dok�adno�� oblicze�.

Wynikiem pracy obu wtyczek jest zbi�r danych, kt�rego definicja sk�adowana jest
w wewn�trznej bazie systemu.

\subsection{Definiowanie zbioru atrybut�w}
Algorytmy generowania drzew decyzyjnych pracuj� na specjalnie
przygotowanych danych -- atrybutach. Atrybuty okre�laj� pewne cechy 
danych wej�ciowych i w przyj�tym w systemie podej�ciu (\ref{lab:knowledge})
bazuj� na zbiorach danych. St�d wtyczka tworz�ca zbi�r atrybut�w -- \emph{AttribiteSetCreator},
kt�ry s�u�y do generowania drzewa decyzyjnego, operuje na tym samym �r�dle danych,
co wtyczka tworz�ca zbi�r danych.

\begin{figure}[!hb]
	\centering
		\includegraphics[scale=0.50]{img/experiment/attrset_settings.png}
	\caption{Konfiguracja zbioru atrybut�w}
	\label{fig:attrset_settings}
\end{figure}

Wtyczka ta pozwala na okre�lenie, kt�re cechy obiekt�w znajduj�cych si�
w zbiorze danych, maj� by� brane pod uwag� podczas budowy drzewa decyzyjnego.
Nale�y okre�li� typ atrybut�w (liczbowy, napisowy, wyliczeniowy) oraz to,
czy dany atrybut jest atrybutem wej�ciowym (wp�ywaj�cym na decyzj�),
czy wyj�ciowym (okre�laj�cym decyzj�).

Efektem uruchomienia wtyczki jest utworzenie i zesk�adowanie zbioru atrybut�w,
odpowiadaj�cego podanej konfiguracji.

\subsection{Generowanie drzewa decyzyjnego}

Do generacji drzew decyzyjnych wykorzystane algorytmy dostarczane wraz z 
\emph{Weka}. W tym celu w ramach platformy zosta� napisany modu� dostarczaj�cy wsparcie dla 
Weki oraz wtyczka, kt�ra korzystaj�c z tego modu�u pozwala na wyb�r algorytmu, 
uruchomienie go i zapisanie wyniku w formacie \emph{Salomona}. \emph{Salomon} posiada w�asne 
\emph{API} przeznaczone do operacji na drzewach. S�u�y do tego g��wnie interfejs 
\emph{ITreeManager}.(zapis, odczyt, modyfikacja itd.). Modu� wspieraj�cy Weke ma za 
zadanie nie tylko t�umaczy� drzewa decyzyjne do wewn�trznego formatu Salomona, 
ale r�wnie� konwertowa� atrybuty z formatu \emph{Salomona} do postaci zgodnej z \emph{Weka}. 
\emph{Weka} dostarczana jest jako dodatkowy plik 'weka.jar'. \emph{Weka} posiada bardziej 
restrykcyjn� licencje (\emph{GPL}), co powoduje, i� Salomon dystrybuowany wraz z 
integracj� musi by� wykorzystywany zgodnie z zasadami tej licencji.

Dla cel�w eksperymentu zosta�a napisana wtyczka \emph{WekaTreeGenerator}, kt�ra 
korzystaj�c z wcze�niej om�wionego modu�u integracji w \emph{Weka}, generuje drzewo 
decyzyjne i zapisuje go do bazy.

Konfiguracja sk�ada si� z:
\begin{enumerate}
  \item wyboru algorytmu �- w eksperymencie zosta� u�yty algorytm \emph{J48}
  \item wyb�r atrybut�w -- opis danych
  \item danych wej�ciowych dla algorytmu
  \item nazwy pod kt�ra zostanie zapisane nowo utworzone drzewo decyzyjne
\end{enumerate}

\begin{figure}[H]
	\centering
		\includegraphics[scale=0.60]{img/experiment/tree_generator_settings.png}
	\caption{Konfiguracja generacji drzewa decyzyjnego}
	\label{fig:tree_generator_settings}
\end{figure}

Wtyczka korzystaj�c z danych trenuj�cych (\emph{DataSet}) oraz opisu atrybut�w 
(\emph{AttributeSet}) transformuje dane (\emph{DataSet}) z wej�ciowej bazy dany do postaci 
zrozumia�ej przez algorytm. Na ko�cu wyniki uruchomienia Weki przekszta�cany 
jest do wewn�trznej reprezentacji drzewa decyzyjnego, u�ywanego przez \emph{Salomona}.

\emph{Weka} zwraca drzewo w formie tekstowej. Opis ten mo�na zobaczy� w otwieraj�c 
okno rezultatu dla omawianej wtyczki (rys)

\begin{figure}[H]
	\centering
		\includegraphics[scale=0.60]{img/experiment/tree_generator_results.png}
	\caption{Konfiguracja losowego zbioru danych}
	\label{fig:tree_generator_results}
\end{figure}

\subsection{Wizualizacja drzewa decyzjnego}
Ostatnim etapem eksperymentu jest wizualizacja nowo powsta�ego drzewa. W tym 
celu zosta�a napisana wtyczka \emph{TreeVis}, kt�ra tworzy graficzn� reprezentacje 
drzewa decyzyjnego. W ustawianiach dla wtyczki podajemy nazw� drzewa,  kt�re 
chcemy z wizualizowa�. Po uruchomieniu mo�emy obejrze� drzewo w postaci grafu. 
Do wizualizacji zosta�a u�yta biblioteka (\emph{Jung}). Przy ka�dym w�le drzewa 
wy�wietlana jest nazwa atrybutu. Na ka�dej z kraw�dzi przypisana jest warto�� 
atrybutu rodzica, kt�ra powoduje przej�cie do w�z�a potomnego. Dla li�ci drzewa 
opr�cz nazwy atrybutu wy�wietlana jest r�wnie� jego warto��.

\begin{figure}[H]
	\centering
		\includegraphics[scale=0.60]{img/experiment/tree_vis_settings.png}
	\caption{Konfiguracja wtyczki wizualizuj�cej drzewo decyzyjne}
	\label{fig:tree_vis_settings}
\end{figure}

\begin{figure}[H]
	\centering
		\includegraphics[scale=0.60]{img/experiment/tree_vis_results.png}
	\caption{Wizualizacja drzewa decyzyjnego}
	\label{fig:tree_vis_results}
\end{figure}


\section{Wyniki eksperymentu}
% - screen z przebiegu
% - opis wynikow (z narysowaniem drzewa)
% - udalo sie zrealizowac cele (uzycie roznych wtyczek i skonfigurowanie ich tak, by uzyskac jakis tam wynik)



\begin{podsumowanie}
%a) Celem pracy by�o, zrobiono, zrealizowano, zaproponowano
%b) Najwa�niejsze osi�gni�cia. Nawi�za� do rozdzia�u drugiego
%c) Plany rozwoju

Celem pracy by�o, om�wienie, zaproponowanie architektury,
oraz dostarczenie przyk�adowej implementacji indukcyjnej bazy
danych jako platformy uczenia maszynowego. 

Dalsze prace nad rozwojem system Salomon powinny koncentrowa� si�
na dodanie wsparcia dla pracy w �rodowisku rozproszonym.




%W pracy zaprezentowano system rozproszonych obliczen� ewolucyjnych bazuja�cy
%na stadnym modelu ewolucji. Koncepcj�e system oparto na analogiach do proces�w zachodza
%�cych w populacjach organizm�w z�ywych.

\end{podsumowanie}
%%
\mgrclosechapter
%%
%%
%% ======== DODATKI ========
%%
%%
%% ======== BIBLIOGRAFIA ========
%%
\bibliography{biblio}
%\begin{thebibliography}{99}
\bibitem {bib1} {Michalski, Kaufman, Pietrzykowski, Sniezynski,
    Wojtusiak, Sharma, Seeman, Fischthal, Alkharouf, White, Draminski,
    Glowinski, ''Inductive Databases and Knowledge Scouts''}

\bibitem {bib2} {Kaufman, K. and Michalski, R. S., ''The Development
    of the Inductive Database System VINLEN: A Review of Current
    Research,'' International Intelligent Information Processing and
    Web Mining Conference, Zakopane, Poland, 2003}

\bibitem {bib3} {Michalski, R. S. and Kaufman, K., ''Data Mining and
    Knowledge Discovery: A Review of Issues and a Multistrategy
    Approach,'' Machine Learning and Data Mining: Methods and
    Applications, R. S. Michalski, I. Bratko and M.  Kubat (Eds.), pp.
    71-112, London: John Wiley \& Sons, 1998}

\bibitem {bib4} {R. S. Michalski, ''Knowledge Mining and Inductive
    Databases: An Emerging New Research Direction'', School of
    Computational Sciences, George Mason University, 2004}
    
%\bibitem {weka} Weka -- \href{http://www.cs.waikato.ac.nz/ml/weka}{http://www.cs.waikato.ac.nz/ml/weka}

%\bibitem {yale} YALE -- \href{yale.cs.uni-dortmund.de}{yale.cs.uni-dortmund.de}

\end{thebibliography}
%%
%% ======== DODATKOWE ELEMENTY PRACY (nieobowi�zkowe) ======== 
%%
\printindex  
%%

\end{document}
