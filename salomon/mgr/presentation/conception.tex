\section{Za�o�enia i koncepcja platformy}

\frame{
	\frametitle{Analiza wymaga�}
	\begin{itemize}		
		\item Przyjazny interfejs u�ytkownika
		\item Dobrze zdefiniowany interfejs programistyczny
		\item Odizolowanie algorytm�w od szczeg��w implementacyjnych plaftormy
		\item Sp�jna reprezentacja wiedzy dla wszystkich algorytm�w
		\item Komponentowa architektura
		\item Wykorzystanie gotowych implementacji algorytm�w
		\item Otwarta licencja \emph{(LGPL)}
	\end{itemize}		
}

\frame{
	\frametitle{G��wne za�o�enia}
	\begin{itemize}
			\item Koncepcja zadaniowo�ci			
			\begin{itemize}
				\item Zadanie -- atomowa jednostka reprezentuj�ca obliczenia
				\item Tworzenie powi�za� mi�dzy zadaniami
			\end{itemize}
    	\item Budowa komponentowa
    	\item Otwarto�� architektury
    	\item Niezale�no�� od �rodowiska wdro�enia
    	\item Prostota u�ytkowania
    \end{itemize}
}

