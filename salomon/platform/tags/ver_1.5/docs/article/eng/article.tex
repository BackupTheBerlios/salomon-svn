%\documentclass[11pt,a4paper,twocolumn]{article}
\documentclass{svmult}

\usepackage{makeidx}         % allows index generation
\usepackage{graphicx}        % standard LaTeX graphics tool
                             % when including figure files
\usepackage{multicol}        % used for the two-column index
\usepackage[bottom]{footmisc}% places footnotes at page bottom

%\usepackage{msk}
\usepackage[english]{babel}
\usepackage[cp1250]{inputenc}
\usepackage[OT4]{fontenc}
%\usepackage{graphicx}
%\usepackage{indentfirst}
\usepackage[urlcolor=black,colorlinks=false,bookmarks=true,bookmarksnumbered=true]{hyperref}


\title{\emph{Salomon} -- %komponentowa architektura indukcyjnej bazy
  %danych jako platformy uczenia maszynowego}
  Component-based Architecture of Inductive Database as a Machine Learning Platform}

\author{
 Krzysztof Rajda\inst{1}
 Nikodem Jura\inst{1} %\and
 Marek Kisiel-Dorohinicki\inst{1} %\and
 Bart�omiej �nie�y�ski\inst{1}}
\authorrunning{Salomon}
\titlerunning{Salomon}

\institute{AGH University of Science and Technology, Institute of
Computer Science, Krak\'ow, Poland\\
\href{mailto:nico@icslab.agh.edu.pl}{nico@icslab.agh.edu.pl},
\href{mailto:krzysztof@rajda.name}{krzysztof@rajda.name},
\href{mailto:doroh@agh.edu.pl}{doroh@agh.edu.pl},
\href{mailto:sniezyn@agh.edu.pl}{sniezyn@agh.edu.pl}}


\begin{document}

\maketitle

\begin{abstract}
%Nowym kierunkiem ��cz�cym technologie baz danych z nowoczesnymi
%metodami indukcyjnego generowania wiedzy s� indukcyjne bazy
%danych.
Inductive database is a new direction which merges database
technology with inductive methods of knowledge generation.
% Po��czenie takie pozwala na jednolity dost�p do wiedzy
%zawartej w~danych bezpo�rednio i wiedzy wygenerowanej z tych
%danych za pomoc� algorytm�w uczenia maszynowego.
This merging allows to access to the knowledge and data in a
consistent manner, e.g. using machine learning algorithms.
% W~niniejszej
%pracy przedstawiona zosta�a architektura i~wybrane aspekty
%implementacji platformy \emph{Salomon}, komponentowej realizacji
%indukcyjnej bazy danych.
The architecture and chosen aspects of implementation of the
inductive database \emph{Salomon} platform have been presented in
this article.
%Dzi�ki budowie modu�owej uda�o si�
%uzyska� du�o lepsz� elastyczno�� systemu w~stosunku do innych,
%zbli�onych rozwi�za�.
Because of the modular structure, a good flexibility in comparison
to other similar systems is gained.

\end{abstract}


\section{Wprowadzenie}
\frame{
	\frametitle{Indukcyjne bazy danych}
		\begin{columns}[t]
			\begin{column}{0.5\textwidth}
				\begin{center}
					\includegraphics[width=\textwidth]{img/knowledge_mining.png}
				\end{center}
			\end{column}
			\begin{column}{0.5\textwidth}
				\begin{idatabase}
					\begin{itemize}
						\item Naturalne rozwini�cie system�w bazodanowych
						\item Udzielaj� odpowiedzi\\ nie tylko bezpo�rednio\\ na podstawie danych
						\item Pozwalaj� na zsyntezowanie wiedzy wygenerowanej poprzez indukcyjne wnioskowanie z~fakt�w i~wcze�niejszej wiedzy
    			\end{itemize}
    		\end{idatabase}
    \end{column}
	\end{columns}
}

\section{Prace pokrewne}

\subsection{VINLEN}
\frame{
	\frametitle{VINLEN}
	\begin{itemize}
		\item Klasyczna realizacja indukcyjnej bazy danych -- mechanizmy wnioskowania indukcyjnego zintegrowane z relacyjnymi operatorami bazodanowymi
		\item Integracja oparta na nowych operatorach generowania wiedzy -- \emph{KGO} (Knowledge Generation Operators)
		\item Synteza i zarz�dzanie wiedz� przez wyspecjalizowanych agent�w (\emph{Scauts})
		\item Operowanie na danych, wiedzy i algorytmach uczenia maszynowego za pomoc� \emph{KGL-1} (\emph{Knowledge
Generation Language})
		\item Sk�adowanie wiedzy wraz z danymi w relacyjnej bazie danych -- \emph{KQL} jako j�zyk zapyta�
		\item Graficzny interfejs u�ytkownika
	\end{itemize}	
}

\subsection{Weka}
\frame{
	\frametitle{Weka}
	\begin{itemize}
		\item Narz�dzie do prowadzenia eksperyment�w zwi�zanych z~uczeniem maszynowym
		\item Dane zapisane w wielu formatach przetwarzane przez \emph{filtry}
		\item Implementacja wielu r�nych rodzaj�w algorytm�w
		\begin{itemize}
			\item klasyfikuj�cych
			\item klastruj�cych
			\item \emph{apriori}
		\end{itemize}
		\item Przyjazny i rozbudowany graficzny interfejs u�ytkownika
		\begin{itemize}
			\item wczytywanie danych do systemu
			\item wizualizacja danych
			\item definiowanie graf�w przep�ywu danych
		\end{itemize}
	\end{itemize}	
}

\subsection{YALE}
\frame{
	\frametitle{YALE}
	\begin{itemize}
		\item Narz�dzie do prowadzenia eksperyment�w zwi�zanych z~uczeniem maszynowym
		\item Mo�liwo�� tworzenia ,,�a�cuch�w'' z�o�onych z r�nych problem�w		
		\begin{itemize}
			\item prze�roczysto�� danych -- r�ne formaty obs�ugiwane przez �rodowisko
			\item �atwo�� ��czenia operator�w -- szerokie zastosowania
		\end{itemize}
		\item R�norodno�� operator�w -- klasyfikuj�ce, klastruj�ce, asocjacyjne, oparte o drzewa decyzyjne
		\item Mo�liwo�� importu algorytm�w z~\emph{Weki}
	\end{itemize}	
}
\section{Koncepcja platformy Salomon}

\subsection{Koncepcja}
\frame{
	\frametitle{Koncepcja}
	\begin{itemize}
		\item Odpowied� na ograniczenia systemu \emph{VINLEN}.
		\item Skalowalno�� jako jeden z g��wnych aspekt�w architektury.
		\item Mo�liwo�� rozpraszania oblicze� dzi�ki rozbiciu procesu obliczeniowego na pojedyncze zadania.
	\end{itemize}		
}

\subsection{G��wne za�o�enia}
\frame{
	\frametitle{G��wne za�o�enia}
	\begin{itemize}
    	\item Budowa komponentowa.
    	\item Otwarto��. 
    	\item Niezale�no�� od �rodowiska wdro�enia.
    	\item Mo�liwo�� rozpraszania oblicze�.
    	\item Prostota u�ytkowania.
    \end{itemize}
}


\section{Architecture of \emph{Salomon}}

System \emph{Salomon} consists of running instances of platform. 
They work together in a shared environment and communicate with each other
to exchange gathered knowledge and tasks to be executed.
Currently system works in client-server architecture, 
it means one instance of platform acts as a server 
and the rest of the instances performs the tasks passed from the server
and return the result of computation.
Server can also perform computation regardless of the clients.

%\section{Architektura systemu \emph{Salomon}}
%
%System \emph{Salomon} tworzy zbi�r uruchomionych instancji platformy,
%dzia�aj�cych we wsp�lnym �rodowisku i komunikuj�cych si� ze sob�
%w~celu wymiany wiedzy oraz przekazania zada\'n do wykonania. Na obecnym etapie
%system pracuje w modelu klient-serwer, tj.\ jedna instancja platformy
%pe�ni rol� zarz�dcy, a~pozosta�e wykonuj� obliczenia przez ni�
%zlecone i~zwracaj� jej wyniki. Sama instancja zarz�dzaj�ca r�wnie�
%mo�e wykonywa� obliczenia.

\begin{figure}[!ht]
    \centering
        \includegraphics[width=\linewidth]{img/architecture.png}
    \caption{Architekura systemu \emph{Salomon}}
    \label{fig:architecture}
\end{figure}

The figure \ref{fig:architecture} demonstrates current architecture of Salomon system.
The following elements can be distinguished:
\begin{description}
	\item[Core] is an integrating element of the system. 
	It enables communication between the rest parts of the system.
	\item[User interface] -- the group of modules responsible for interaction with a user.
	\item[Knowledge manager] enables storing and managing varied types of knowledge 
	e.g. decision trees, clusters, rules etc.
	For each kind of knowledge the separate implementation is defined. 
	It is enables plug-ins the interface to manage specific type of knowledge.
	It makes using knowledge easier and separates plug-ins from the implementation details 
	e.g. storing the knowledge in database.
	\item[Data manager] is an abstraction layer that makes Salomon be independent 
	from varied implementation of relational database systems.
	\item[Communication module] enables computation distribution.
	\item[Plug-ins] provide algorithms that search for knowledge, process it, 
	present the result of computation etc.
	Execution of algorithms is performed in isolated environment. 
	During their work plug-ins have an access to knowledge, data and the user interface.
	\item[Data] stored in a single location may be used by many instances of the system.
	\item[Knowledge] is gathered and stored by each instance of the system 
	along with the data needed by the system such as information about plug-ins installed, 
	created projects, data sources etc.
	\item[Task manager] is responsible for executing and managing task execution.
\end{description}

%Rysunek \ref{fig:architecture} przedstawia obecn� architektur�
%systemu Salomon. Mo�na w niej wyr�ni� nast�puj�ce elementy:
%\begin{description}
%\item[J�dro systemu] to element integruj�cy pozosta�e modu�y. Zapewnia
% komunikacj� pomi�dzy poszczeg�lnymi elementami systemu.
%\item[Interfejs u�ytkownika] -- tworzy pakiet modu�\'ow
% odpowiedzialnych za komunikacj� z~u�ytkownikiem.
%\item[Modu� zarz�dzaj�cy wiedz�] kt\'orego wyodr�bnienie pozwala
% \emph{Salomonowi} na przechowywanie i operowanie na r\'o\.znych
% rodzajach wiedzy np.\ drzewach decyzyjnych, klastrach, regu�ach itp.
% Dla ka�dego rodzaju wiedzy definiowana jest osobna implemetacja, kt�ra
% odpowiada za udost�pnienie wtyczkom interfejsu do zarz�dzania danym
% rodzajem wiedzy. U�atwia to operowanie na wiedzy oraz odseparowuje
% wtyczki od szczeg��w implementacyjnych pozwalaj�cych np.\ na
% przechowywanie wiedzy w bazie danych.
%\item[Modu� zarz�dzaj�cy danymi] stanowi warstw� abstrakcji dost�pu do
%  danych. Pozwala na uniezale�nienie reszty systemu od
%  r�nych implementacji system�w bazodanowych.
%\item[Modu� komunikacyjny] umo�liwiaj�cy rozproszenie oblicze�.
%\item[Wtyczki] s� odpowiedzialne za dostarczenie algorytm�w
% wyszukuj�cych, przetwarzaj�cych wiedz�, za prezentacje wynik�w itp.
% Wykonanie algorytm�w przez nich dostarczonych odbywa si� w izolowanym
% �rodowisku.  W~trakcie pracy wtyczki maj� dost�p do wiedzy, do danych
% i~do interfejsu u�ytkownika.
%\item[Dane] z pojedynczego �r�d�a wykorzystywa\'c mo�e wiele instancji
% systemu.
%\item[Wiedza] gromadzona i przechowywana jest przez ka�d� instancj�
% wraz danymi potrzebnymi w~trakcie dzia�ania systemu, takimi jak
% informacje o zainstalowanych wtyczkach, zdefiniowane projekty, �r�d�a
% danych itp
%\item[Kontroler zada�] jest odpowiedzialny za uruchamianie i
% kontrolowanie wykonywania poszczeg�lnych zada�.
%\end{description}


In distributed systems there are problems with keeping interoperability
of individual instances -- the may have not only different version of a system 
but also the implementation. For this reason there was necessary to define
clear layers and interfaces between them.
The most important layers that we can distinguish in \emph{Salomon} are:

%Poniewa\.z w systemach rozproszonych mog� wyst�pi� problemy z
%utrzymaniem zgodno�ci (ang.\ \emph{interoperability}) nie tylko ze
%wzgl�du na wersje systemu, ale tak�e na implementacje, st�d potrzeba
%zdefiniowania jasno okre�lonych warstw i interfejs�w pomi�dzy nimi.
%Do najwa�niejszych warstw, jakie mo�emy wyr�ni� w systemie
%\emph{Salomon} nale��:


\begin{enumerate}
	\item \emph{Communication with the external data sources.}
	Communication is realized by specialized components that hide differences between varied data sources.
	They provide the interface that is used by plug-ins to extract data and its structure.
	
	\item \emph{Communication between varied instances of Salomon.}
	Currently the communication is realized using \emph{RMI} technology. 
	The system is homogenic, so using \emph{RMI} seems to be the natural choice.
	In case of appearance of the new implementations it may be necessary 
	to implement communication based on more universal technologies e.g. \emph{CORBA} as well.

	\item \emph{Communication with plug-ins.}
	The group of specialized classes provide data from external data source for plug-ins 
	and enable them performing operations on knowledge. Plug-ins are completely separated from
	the external environment and this layer is the one place when the knowledge exchange 
	between plug-ins and the environment takes place.
	
	\item \emph{User interface.} 
	System \emph{Salomon} may be used in varied contexts, such as interaction with the user, 
	providing services in distributed network environment or as a library used by another program.
	All this forces very strong separation between user interface and the rest of the system.
	
\end{enumerate}

%\begin{enumerate}
%    \item \emph{Komunikacja z zewn�trznymi �r�d�ami danych.}
%    Komunikacja odbywa si� przez specjalne komponenty, kt�re ukrywaj� r�nice pomi�dzy r�nymi �r�d�ami danych.
%    Udost�pniaj� one specjalny interfejs, za pomoc� kt�rego wtyczki mog� pobra� dane oraz struktur� bazy.
%
%    \item \emph{Komunikacja mi�dzy r�nymi instancjami Salomona.}
%    W obecnej wersji systemu do komunikacji wykorzystywana  jest technologia \emph{RMI}.
%    Na obecnym etapie system jest homogeniczny, wi�c naturalnym wydaje by�
%    u�ycie w�a�nie \emph{RMI}. W przypadku pojawianie si� innych implementacji, konieczne mo�e by� zaimplementowanie
%    r�wnie� komunikacji w~oparciu o bardziej uniwersalne technologie jak np. \emph{CORBA}.
%
%    \item \emph{Komunikacja z wtyczkami.}
%    Zestaw specjalnych klas odpowiada za dostarczenie wtyczkom danych z zewn�trznego �r�d�a danych oraz za umo�liwienie operacji na wiedzy. Wtyczki s� ca�kowicie odseparowane od zewn�trznego �rodowiska i~warstwa ta stanowi jedyne miejsce wymiany wiedzy pomi�dzy nimi a~�rodowiskiem.
%
%    \item \emph{Interfejs u�ytkownika.}
%System \emph{Salomon} mo�e by� wykorzystywany w~r�nych kontekstach, takich jak na przyk�ad praca z~u�ytkownikiem,
%udost�pnianie us�ug w~rozproszonym �rodowisku sieciowym czy jako biblioteka wykorzystywana w innym programie.
%To wymusza bardzo mocne oddzielenie interfejsu u�ytkownika od reszty systemu.

%\end{enumerate}

\section{Tests of the system}
%\section{Wykonane testy systemu}

Salomon was tested using freely available data sets from UCI
repository~\cite{uci}. Three plug-ins were use.

%System \emph{Salomon} zosta� przetestowany na powszechnie dost�pnych
%zbiorach danych do testowania algorytm�w uczenia maszynowego
%UCI~\cite{uci}. Przetestowane zosta�y wtyczki zwi�zane z uczeniem
%drzew decyzyjnych.


The aim of the first plug-in was to configure the data processing. It enables
choosing both major and minor characteristics taken into account during building
decision tree. The choice is made with the use of a graphical user interface.
The second plug-in is the actual processing component -� it provides the
realization of the \emph{ID3} algorithm. The last plug-in is responsible for
visualization of the tree created during computation.

%Zadaniem pierwszej z nich jest konfiguracja procesu przetwarzania
%danych. Pozwala ona na wyb�r cechy, wzgl�dem kt�rej tworzone jest
%drzewo decyzyjne oraz cech, kt�re s� brane pod uwag� przy jego
%budowie. Wyb�r ten odbywa si� przy u�yciu graficznego interfejsu
%u�ytkownika. Druga wtyczka jest w�a�ciw� wtyczk� obliczeniow� --
%jest to realizacja algorytmu ID3. Ostatnia wtyczka ma na celu
%wizualizacj� utworzonego drzewa.

These test were quite simple. Currently more complex tasks are
being addressed. One of them is described in the next section.

\section{Case study}
%Rozwa�my sytuacje, kiedy zamierzamy stworzy� system wspomagaj�cy onkolog�w w 
%diagnostyce nowotworowej.Do dyspozycji mamy dane z wielu szpitali rozproszonych 
%po ca�ej Polsce.
Let us consider situations in which we are going to create a system aiding 
oncologists in tumor diagnosis. We have data from  
many hospitals distributed all over entire Poland.
%Ze wzgl�d�w wydajno�ciowych oraz bezpiecze�stwa (szpitale nie 
%wyra�aj� zgody udost�pnianie ich danych innym plac�wk�) nie mo�liwy jest 
%jednoczesny dost�p do wszystkich baz. Naszym celem stworzenie regu� 
%wspomagaj�cych lekarzy przy diagnostyce nowotwor�w. W tym celu przej�li�my 
%nast�puj�ce podej�cie.
Because of the efficiency and the safety constraints (hospitals are not wishing to share their data with other institutions) a 
simultaneous access to all databases is not possible. Our purpose is to create
a set of rules, which aid doctors at the diagnosis of cancer. To achieve 
this purpose we adopted the following approach.
%Szukamy generalnych regu� dla wszystkich przypadk�w 
%(zgodnych z danymi w ka�dej bazie) oraz regu� najbardziej pasuj�cych dla danej 
%plac�wki. Dodatkowo zale�y nam na grupowanie przypadk�w ze wzgl�du na ich 
%zgodno�� do og�lnych regu� (najbardziej, �rednio i najmniej zgodne) i dla 
%ka�dej z tych grup znalezienie bardziej specyficznych regu�. 
We search for general rules for all coincidences (according to 
data in every base) and of rules for the most fitting every institution. 
Additionally, we focus on grouping cases on account of their
agreement to general rules (the most, on average and least agreeable) and for each of these groups finding more unique rules.
% Takie pogrupowanie pozwoli na uproszenie regu� (stan� si�
%czytelniejsze 
%dla lekarzy) dla ka�dej takiej grupy, co pozwoli analitykowi na ewentualne 
%sterowanie procesem uczenia. Dodatkowo chcieliby�my mie� dost�p do danych 
%statystycznych, dla poszczeg�lnych plac�wek (por�wnanie z reszt�).
Grouping coincidences allows to simplify rules (will become more readable 
for doctors) for each such group, which will allow the analyst to 
potentially steer the process of learning. Additionally, we would like to have 
access to statistical data, for individual institutions (comparison to the 
rest).
%Wp�yw r�nych czynnik�w w zale�no�ci o po�o�enia geograficznego, oraz wielko�� 
%poczeg�lnych gr�p. Takie dane nie tylko u�yteczne byby�y dla lekarzy czy os�b 
%steruj�cych procesem odkrywania widzy, ale mog�by by� r�wnie� wykorzystane jako 
%dane wej�ciowe dla algorytm�w uczenia maszynowego.
The influence of different factors in the relation against locations geographical, and size of particular group. 
Such data could be useful not only for doctors or people responsible for steering the process of learning, but they could be used
as the input for algorithms of machine learning.

%%%%%%%%%%%%%%%%%%%%%%%%%%%%%%%
%Przedstawiamy spos� realizacji takiego zadania z wykorzystaniem systemu
%\emph{Salomon}. Obecna wersja nie posiada zaimplementowanych wszystkich
%niezb�dnych funkcjonalno�ci, jednak wszystkie one powinny pojawi� si� w kolejnym
%wydaniu.
We present the way to the realization of such an objective by using 
the system \emph{Salomon}. The current version lacks the implementation of all 
essential functionalities, however they all should available at the next
edition.

%%%%%%%%%%%%%%%%%%%%%%%%%%%%%%%%
%Zgodnie z za�o�eniami zadania, nie mo�liwa jest wymiana danych o pacjentach 
%mi�dzy plac�wkami. Wymaga�oby to stworzenia pot�nego centrum sk�aduj�cego 
%wszystkie dane, albo dost�pu zdalnego do tych danych dla algorytm�w, co jednak 
%z racji ograniczonej przepustowo�ci sieci bardzo spowolni�oby obliczenia.
According to the principles of the objective, an exchange of data about 
patients is not possible among institutions. It would require creating a 
vast centre for storing all data or the remote access to these data for algorithms.
However, a remote access could reduce the speed of computation because of the 
network capacity.
%Dodatkowo tak wielka ilo�� danych jest nie mo�liwa do przeanalizowania w 
%sensownym czasie. Je�li w naszych przyk�adowym zastosowaniu bezpiecze�stwo 
%danych nie jest spraw� krytyczn�, to jednak mo�na sobie wyobrazi� wiele 
%przyk�adowych zastosowa�, dla kt�rych ka�dej plac�wce zale�a�oby na 
%nieudost�pnianiu swoich danych.
Additionally, such amount of data is not possible to be analyzed in a 
sensible time. In our demonstration applying the data security it is not a 
critical matter, however it is possible to imagine a lot 
of example usages, in which every institution cares about unauthorized access 
to its data.

%%%%%%%%%%%%%%%%%%%%%%%%%
%Dlatego naturalnym rozwi�zaniem jest 
%wyszukiwanie wiedzy w ka�dej plac�wce z osobna, a przesy�anie jedynie wiedzy 
%mi�dzy pomi�dzy poszczeg�lnymi w�z�ami. Dlatego w ka�dej w�le zostanie 
%zainstalowany agent systemu. Obrazek \ref{fig:example} przedstawia koncepcje
%przyk�adowej architektury.
Therefore, searching for the knowledge at every institution individually and 
exchanging knowledge only between individual institution, is a natural solution.
In every node \emph{Salomon Agent} is installed agent. Picture \\ ref{fig:example}
is presenting conceptions of demonstration architecture.

\begin{figure}[!ht]
    \centering
        \includegraphics[width=0.8\linewidth]{img/hospital_example.png}
%    \caption{Architekura systemu \emph{Salomon}}
    \caption{The architecture of the example}
    \label{fig:example}
\end{figure}

%System reagowa� b�dzie na zdarzenia:
The system will react to events:

\begin{enumerate}
%  \item Bazda danych zosta�a zmieniona wi�cej ni� 20\% W takim przypadku 
%  zostan� podj�te nast�puj�ce kroki:
  \item The database was changed more than the 20\% in such case the 
  following actions will be undertaken:
	  \begin{enumerate}
%		\item Dost�pne regu�y i klastry zostan� sprawdzone na nowych danych i 
%		ewentualnie zostan� ulepszone
		\item Aviable rules and clusters will be checked on new data and ,if necessary,
%		\item Zostan� wygenerowane nowe regu�y i klastry i por�wnane z najlepszymi 
%		obecnie dost�pnymi
		\item New rules and clusters will be generated and compared with the best
		avialiable ones
%		\item Zostan� wygenerowane nowe dane statystyczne
		\item New statistic data will be generated
%		\item Je�li rezultatem powy�ych krok�w b�dzie uzyskanie lepszej wiedzy, 
%		zostanie ona rozes�ana do reszty w�z��w
		\item If a better knowledge is a result of the above-mentioned steps, it will be
		sent out up to other nodes
      \end{enumerate}
%	\item Inny w�ze� wygenerowa� now� wiedz�
	\item The different node generated the new knowledge
		\begin{enumerate}
%		  \item Wiedza z innego agenta zostanie przetestowana na lokalnych danych
		  \item The knowledge in a different agent will be tested on local data
%		  \item Je�li wynik test�w jest zadawalaj�cy, wiedza ta zostanie ulepszona
%		  korzystaj�c z lokalnych danych
		  \item If the result of tests is satisfactory, the knowledge will be improved
		  using local data  %!! nie wiem czy cos takiego jest mozliwego
		  % wiedza z innego wezla zostanie sumowana z lokana wiedza po czym rezultat
		  % zostanie oceniony
%		  \item W przypadku otrzymania lepszych reg�, klastry zostana zaktualizowane
		  \item In case of obtained better rules, clusters will be updated
		  %\item Wiedza statystyczna zostanie zaktualizowana
		  \item The statistical knowledge will be updated
%		  \item Podobnie jak w przyadku pierwszego zda�enia, je�li powy�sze kroki
%			wygeneruj� dostaniecznie lepsz� wiedz�, zostanie ona rozes�ana do pozosta�ych
%			w�z��w.
		  \item As in case of the first event, if above-mentioned steps generate knowledge which is good
		  enough, it will be sent out to another nodes.
		\end{enumerate}
\end{enumerate}

%Dzi�ki takiej architekturze mogliby�my otrzyma� system kt�ry:
Thanks to such architecture, we gained a system, which:
\begin{enumerate}
%  \item Zapewnia bezpiecz��stwo danych dla ka�dego agenta
  \item Provides data security for every agent
%  \item Ilo�� przysy�anych danych jest zminimalizowana
  \item Minimizes the amount of sent data
%  \item System reaguje na zdarzenia, zatem skomplikowane obliczenia wykonywane
%  s� tylko w przypadku powstania nowych danych lub nowej wiedzy
  \item Reacts to events, and so computing occurs only in case of the
  uprising of new data or the new knowledge being generated
%  \item Ka�dy w�ze� opr�cz og�lnej (najbardziej odpowiadaj�cej wszystkim w�z��)
%  wiedzy, posiada tak�e wi�dz� specyficzn� dla swoich danych
  \item Every node apart from general (the most suitable to all nodes) of 
  knowledge, has also knowledges unique to its data
%  \item Mo�liwo�� zastosowania r�nych algorytm�w i r�nej ich konfiguracji dla
%  ka�ego agenta z osobna
  \item Possibility of applying different algorithms and of their different 
  configuration for every agent individually
%  \item System \emph{Salomon} mo�na zainstalowa� w �rodowisko heterogenicznym.
%  Ka�da plac�wka mo�e posiada� inny system bazodanowy (np. inny schemat bazy
%  danych).
%  Algorytm przekszta�ci�by r�nie zapisane dane do postaci jednolitych
%  atrybut�w.
  \item System \emph{Salomon} it is possible to become installed into 
  heterogeneous environments. Every institution can have a different database system (e.g.
  different database schema). By using attributes, the knowledge could be
  distributed among nodes. The algorithm would convert data differently saved to the figure
  of uniform attributes.

%  \item Obliczenia nie s� scentralizowane. W ka�dej chwili mo�na dodawa� nowych
%  i usuwa� stary agent�w.
  \item Calculations are not centralized. At any time, it is possible to add new
  or to remove old of agents.
\end{enumerate}

% obliczenia nie sa scentralizowane w ka�dej chwili mo�nda dodawa� nowe i usuwa�
% stare w�z�y

%Jak wida� na powy�szym przyk��dzie, \emph{Salomon} �wietnie nadaj� si� do
%rozwi�zywania skomplikowanych i rozproszonych oblicze�. Pozwalaj�c przy tym na
%minimalizacje wysi�ku oraz zasob�w. Nowa wersja platformy ma zawiera� te
%udogodnienia, kt�r� s� niezb�dn� do realizacji tego zadania. Jednak potrzebne s�
%tak�e odpowiednie wtyczki, kt�re dostarcza�by odpowiednich algorytm�w.
As can seen on above example, \emph{Salomon} is fit for solving
complicated and distributed computation. It allows for  
minimizations of effort and resources usage. An updated version of
the platform is supposed to contain these conveniences, which are necessary for the 
realization of this objective. However appropriate plugs, which provide
algorithms, are also needed.


\subsection{Dalsze prace}
\frame{
	\frametitle{Dalsze prace}
	\begin{itemize}
		\item Mo�liwo�� rozproszenia oblicze�
    	\item Rozdzielanie zada� -- graf przep�ywu
    	\item Wsparcie dla innych typ�w wiedzy
    	\item Wsparcie dla r�nych zewn�trznych �r�de� danych
    \end{itemize}
}

\begin{thebibliography}{99}
\bibitem {bib1} {Michalski, Kaufman, Pietrzykowski, Sniezynski,
    Wojtusiak, Sharma, Seeman, Fischthal, Alkharouf, White, Draminski,
    Glowinski, ''Inductive Databases and Knowledge Scouts''}

\bibitem {bib2} {Kaufman, K. and Michalski, R. S., ''The Development
    of the Inductive Database System VINLEN: A Review of Current
    Research,'' International Intelligent Information Processing and
    Web Mining Conference, Zakopane, Poland, 2003}

\bibitem {bib3} {Michalski, R. S. and Kaufman, K., ''Data Mining and
    Knowledge Discovery: A Review of Issues and a Multistrategy
    Approach,'' Machine Learning and Data Mining: Methods and
    Applications, R. S. Michalski, I. Bratko and M.  Kubat (Eds.), pp.
    71-112, London: John Wiley \& Sons, 1998}

\bibitem {bib4} {R. S. Michalski, ''Knowledge Mining and Inductive
    Databases: An Emerging New Research Direction'', School of
    Computational Sciences, George Mason University, 2004}
    
%\bibitem {weka} Weka -- \href{http://www.cs.waikato.ac.nz/ml/weka}{http://www.cs.waikato.ac.nz/ml/weka}

%\bibitem {yale} YALE -- \href{yale.cs.uni-dortmund.de}{yale.cs.uni-dortmund.de}

\end{thebibliography}

%wylaczenie numerowania na I stronie
\thispagestyle{empty}


\end{document}
