\subsection{Abbot}

W projekcie chcemy doda� GUI Testy. U�ywamy do tego Abbota(\href{http://abbot.sourceforge.net/}{http://abbot.sourceforge.net/}). Jest to prosty framework pozwalaj�cy na takie w�a�nie testy. Na razie zaimplementowali�my przyk�adowy test, odpalaj�cy okienko Salomona. To podstawowy test pozwalaj�cy stwierdzi� czy program dzia�a czy nie.


\subsubsection{Instalacja}
Program mo�na �ci�gn�� ze strony projektu. Razem z dystrybucj� dostajemy przyjemne GUI do tworzenia tych test�w.
Testy zapisywane s� w XMLu.\\
Nale�y ponadto doda� odpowiednie jar-y do classpath (\texttt{abbot.jar}, \texttt{xalan.jar}, \texttt{xml-apis.jar}).
\subsubsection{Test}
\begin{Verbatim}
<AWTTestScript 
  desc="Skrypt (D:\home\krzychu\dv\ks\test\abbot\salomon.xml)"
  forked="true">
  <launch 
    args="[]"
    class="salomon.engine.Starter" 
    classpath="
      salomon.jar;
      .;
      lib\mini-j2ee.jar;
      lib\log4j-1.2.8.jar;
      lib\firebirdsql.jar"
    method="main"/>
  <wait 
    args="Salomon"
    class="abbot.tester.ComponentTester"
    method="assertFrameShowing"/>
  <terminate/>
</AWTTestScript>
\end{Verbatim}
Jak wida� w te�cie tym sprawdzamy czy okienko si� pokazuje czy te� nie.

\subsubsection{Uruchomienie testu}
\begin{Verbatim}
java -cp lib/abbot.jar 
	junit.extensions.abbot.ScriptFixture test/abbot/salomon.xml
\end{Verbatim}

Po uruchomieniu tego dostajemy output:
\begin{Verbatim}
$ runabbot.bat

D:\home\krzychu\dv\ks>java -cp lib/abbot.jar 
	junit.extensions.abbot.ScriptFixture test/abbot/salomon.xml
.
Time: 38,035

OK (1 test)
\end{Verbatim}

W planach mamy bardziej zaawansowane testy w miar� jak b�dzie si� poszerza� funkcjonalno��.
