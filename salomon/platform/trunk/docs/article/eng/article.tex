%\documentclass[11pt,a4paper,twocolumn]{article}
\documentclass{svmult}

\usepackage{makeidx}         % allows index generation
\usepackage{graphicx}        % standard LaTeX graphics tool
                             % when including figure files
\usepackage{multicol}        % used for the two-column index
\usepackage[bottom]{footmisc}% places footnotes at page bottom

%\usepackage{msk}
\usepackage[english]{babel}
\usepackage[cp1250]{inputenc}
\usepackage[OT4]{fontenc}
%\usepackage{graphicx}
%\usepackage{indentfirst}
\usepackage[urlcolor=black,colorlinks=false,bookmarks=true,bookmarksnumbered=true]{hyperref}


\title{\emph{Salomon} -- %komponentowa architektura indukcyjnej bazy
  %danych jako platformy uczenia maszynowego}
  component architecture of inductive database as a machine learning platform}

\author{Nikodem Jura, Krzysztof Rajda,
 Marek Kisiel-Dorohinicki,
 Bart�omiej �nie�y�ski}
 
\institute{Akademia G�rniczo-Hutnicza, Katedra Informatyki,
Krak�w\\
\href{mailto:nico@icslab.agh.edu.pl}{nico@icslab.agh.edu.pl},
\href{mailto:krzysztof@rajda.name}{krzysztof@rajda.name},
\href{mailto:doroh@agh.edu.pl}{doroh@agh.edu.pl},
\href{mailto:sniezyn@agh.edu.pl}{sniezyn@agh.edu.pl}}


\begin{document}

\maketitle

\begin{abstract}
%Nowym kierunkiem ��cz�cym technologie baz danych z nowoczesnymi
%metodami indukcyjnego generowania wiedzy s� indukcyjne bazy
%danych. 
The new direction which connects database technology with modern inductive
methods of generation knowledge is inductive databases.
% Po��czenie takie pozwala na jednolity dost�p do wiedzy
%zawartej w~danych bezpo�rednio i wiedzy wygenerowanej z tych
%danych za pomoc� algorytm�w uczenia maszynowego.
This kind of  connection allows direct access to knowledge in data consistently
using algorithms of machine learning. 
% W~niniejszej
%pracy przedstawiona zosta�a architektura i~wybrane aspekty
%implementacji platformy \emph{Salomon}, komponentowej realizacji
%indukcyjnej bazy danych. 
The architecture and chosen aspects of implementation of the \emph{Salomon}
platform have been introduced is this article.
%Dzi�ki budowie modu�owej uda�o si�
%uzyska� du�o lepsz� elastyczno�� systemu w~stosunku do innych,
%zbli�onych rozwi�za�.
Owing to the modular structure, we gained much better system flexibility
in comparison to other similar systems.

\end{abstract}


\section{Wprowadzenie}
\frame{
	\frametitle{Indukcyjne bazy danych}
		\begin{columns}[t]
			\begin{column}{0.5\textwidth}
				\begin{center}
					\includegraphics[width=\textwidth]{img/knowledge_mining.png}
				\end{center}
			\end{column}
			\begin{column}{0.5\textwidth}
				\begin{idatabase}
					\begin{itemize}
						\item Naturalne rozwini�cie system�w bazodanowych
						\item Udzielaj� odpowiedzi\\ nie tylko bezpo�rednio\\ na podstawie danych
						\item Pozwalaj� na zsyntezowanie wiedzy wygenerowanej poprzez indukcyjne wnioskowanie z~fakt�w i~wcze�niejszej wiedzy
    			\end{itemize}
    		\end{idatabase}
    \end{column}
	\end{columns}
}

\section{Prace pokrewne}

\subsection{VINLEN}
\frame{
	\frametitle{VINLEN}
	\begin{itemize}
		\item Klasyczna realizacja indukcyjnej bazy danych -- mechanizmy wnioskowania indukcyjnego zintegrowane z relacyjnymi operatorami bazodanowymi
		\item Integracja oparta na nowych operatorach generowania wiedzy -- \emph{KGO} (Knowledge Generation Operators)
		\item Synteza i zarz�dzanie wiedz� przez wyspecjalizowanych agent�w (\emph{Scauts})
		\item Operowanie na danych, wiedzy i algorytmach uczenia maszynowego za pomoc� \emph{KGL-1} (\emph{Knowledge
Generation Language})
		\item Sk�adowanie wiedzy wraz z danymi w relacyjnej bazie danych -- \emph{KQL} jako j�zyk zapyta�
		\item Graficzny interfejs u�ytkownika
	\end{itemize}	
}

\subsection{Weka}
\frame{
	\frametitle{Weka}
	\begin{itemize}
		\item Narz�dzie do prowadzenia eksperyment�w zwi�zanych z~uczeniem maszynowym
		\item Dane zapisane w wielu formatach przetwarzane przez \emph{filtry}
		\item Implementacja wielu r�nych rodzaj�w algorytm�w
		\begin{itemize}
			\item klasyfikuj�cych
			\item klastruj�cych
			\item \emph{apriori}
		\end{itemize}
		\item Przyjazny i rozbudowany graficzny interfejs u�ytkownika
		\begin{itemize}
			\item wczytywanie danych do systemu
			\item wizualizacja danych
			\item definiowanie graf�w przep�ywu danych
		\end{itemize}
	\end{itemize}	
}

\subsection{YALE}
\frame{
	\frametitle{YALE}
	\begin{itemize}
		\item Narz�dzie do prowadzenia eksperyment�w zwi�zanych z~uczeniem maszynowym
		\item Mo�liwo�� tworzenia ,,�a�cuch�w'' z�o�onych z r�nych problem�w		
		\begin{itemize}
			\item prze�roczysto�� danych -- r�ne formaty obs�ugiwane przez �rodowisko
			\item �atwo�� ��czenia operator�w -- szerokie zastosowania
		\end{itemize}
		\item R�norodno�� operator�w -- klasyfikuj�ce, klastruj�ce, asocjacyjne, oparte o drzewa decyzyjne
		\item Mo�liwo�� importu algorytm�w z~\emph{Weki}
	\end{itemize}	
}
\section{Koncepcja platformy Salomon}

\subsection{Koncepcja}
\frame{
	\frametitle{Koncepcja}
	\begin{itemize}
		\item Odpowied� na ograniczenia systemu \emph{VINLEN}.
		\item Skalowalno�� jako jeden z g��wnych aspekt�w architektury.
		\item Mo�liwo�� rozpraszania oblicze� dzi�ki rozbiciu procesu obliczeniowego na pojedyncze zadania.
	\end{itemize}		
}

\subsection{G��wne za�o�enia}
\frame{
	\frametitle{G��wne za�o�enia}
	\begin{itemize}
    	\item Budowa komponentowa.
    	\item Otwarto��. 
    	\item Niezale�no�� od �rodowiska wdro�enia.
    	\item Mo�liwo�� rozpraszania oblicze�.
    	\item Prostota u�ytkowania.
    \end{itemize}
}


\section{Architecture of \emph{Salomon}}

System \emph{Salomon} consists of running instances of platform. 
They work together in a shared environment and communicate with each other
to exchange gathered knowledge and tasks to be executed.
Currently system works in client-server architecture, 
it means one instance of platform acts as a server 
and the rest of the instances performs the tasks passed from the server
and return the result of computation.
Server can also perform computation regardless of the clients.

%\section{Architektura systemu \emph{Salomon}}
%
%System \emph{Salomon} tworzy zbi�r uruchomionych instancji platformy,
%dzia�aj�cych we wsp�lnym �rodowisku i komunikuj�cych si� ze sob�
%w~celu wymiany wiedzy oraz przekazania zada\'n do wykonania. Na obecnym etapie
%system pracuje w modelu klient-serwer, tj.\ jedna instancja platformy
%pe�ni rol� zarz�dcy, a~pozosta�e wykonuj� obliczenia przez ni�
%zlecone i~zwracaj� jej wyniki. Sama instancja zarz�dzaj�ca r�wnie�
%mo�e wykonywa� obliczenia.

\begin{figure}[!ht]
    \centering
        \includegraphics[width=\linewidth]{img/architecture.png}
    \caption{Architekura systemu \emph{Salomon}}
    \label{fig:architecture}
\end{figure}

The figure \ref{fig:architecture} demonstrates current architecture of Salomon system.
The following elements can be distinguished:
\begin{description}
	\item[Core] is an integrating element of the system. 
	It enables communication between the rest parts of the system.
	\item[User interface] -- the group of modules responsible for interaction with a user.
	\item[Knowledge manager] enables storing and managing varied types of knowledge 
	e.g. decision trees, clusters, rules etc.
	For each kind of knowledge the separate implementation is defined. 
	It is enables plug-ins the interface to manage specific type of knowledge.
	It makes using knowledge easier and separates plug-ins from the implementation details 
	e.g. storing the knowledge in database.
	\item[Data manager] is an abstraction layer that makes Salomon be independent 
	from varied implementation of relational database systems.
	\item[Communication module] enables computation distribution.
	\item[Plug-ins] provide algorithms that search for knowledge, process it, 
	present the result of computation etc.
	Execution of algorithms is performed in isolated environment. 
	During their work plug-ins have an access to knowledge, data and the user interface.
	\item[Data] stored in a single location may be used by many instances of the system.
	\item[Knowledge] is gathered and stored by each instance of the system 
	along with the data needed by the system such as information about plug-ins installed, 
	created projects, data sources etc.
	\item[Task manager] is responsible for executing and managing task execution.
\end{description}

%Rysunek \ref{fig:architecture} przedstawia obecn� architektur�
%systemu Salomon. Mo�na w niej wyr�ni� nast�puj�ce elementy:
%\begin{description}
%\item[J�dro systemu] to element integruj�cy pozosta�e modu�y. Zapewnia
% komunikacj� pomi�dzy poszczeg�lnymi elementami systemu.
%\item[Interfejs u�ytkownika] -- tworzy pakiet modu�\'ow
% odpowiedzialnych za komunikacj� z~u�ytkownikiem.
%\item[Modu� zarz�dzaj�cy wiedz�] kt\'orego wyodr�bnienie pozwala
% \emph{Salomonowi} na przechowywanie i operowanie na r\'o\.znych
% rodzajach wiedzy np.\ drzewach decyzyjnych, klastrach, regu�ach itp.
% Dla ka�dego rodzaju wiedzy definiowana jest osobna implemetacja, kt�ra
% odpowiada za udost�pnienie wtyczkom interfejsu do zarz�dzania danym
% rodzajem wiedzy. U�atwia to operowanie na wiedzy oraz odseparowuje
% wtyczki od szczeg��w implementacyjnych pozwalaj�cych np.\ na
% przechowywanie wiedzy w bazie danych.
%\item[Modu� zarz�dzaj�cy danymi] stanowi warstw� abstrakcji dost�pu do
%  danych. Pozwala na uniezale�nienie reszty systemu od
%  r�nych implementacji system�w bazodanowych.
%\item[Modu� komunikacyjny] umo�liwiaj�cy rozproszenie oblicze�.
%\item[Wtyczki] s� odpowiedzialne za dostarczenie algorytm�w
% wyszukuj�cych, przetwarzaj�cych wiedz�, za prezentacje wynik�w itp.
% Wykonanie algorytm�w przez nich dostarczonych odbywa si� w izolowanym
% �rodowisku.  W~trakcie pracy wtyczki maj� dost�p do wiedzy, do danych
% i~do interfejsu u�ytkownika.
%\item[Dane] z pojedynczego �r�d�a wykorzystywa\'c mo�e wiele instancji
% systemu.
%\item[Wiedza] gromadzona i przechowywana jest przez ka�d� instancj�
% wraz danymi potrzebnymi w~trakcie dzia�ania systemu, takimi jak
% informacje o zainstalowanych wtyczkach, zdefiniowane projekty, �r�d�a
% danych itp
%\item[Kontroler zada�] jest odpowiedzialny za uruchamianie i
% kontrolowanie wykonywania poszczeg�lnych zada�.
%\end{description}


In distributed systems there are problems with keeping interoperability
of individual instances -- the may have not only different version of a system 
but also the implementation. For this reason there was necessary to define
clear layers and interfaces between them.
The most important layers that we can distinguish in \emph{Salomon} are:

%Poniewa\.z w systemach rozproszonych mog� wyst�pi� problemy z
%utrzymaniem zgodno�ci (ang.\ \emph{interoperability}) nie tylko ze
%wzgl�du na wersje systemu, ale tak�e na implementacje, st�d potrzeba
%zdefiniowania jasno okre�lonych warstw i interfejs�w pomi�dzy nimi.
%Do najwa�niejszych warstw, jakie mo�emy wyr�ni� w systemie
%\emph{Salomon} nale��:


\begin{enumerate}
	\item \emph{Communication with the external data sources.}
	Communication is realized by specialized components that hide differences between varied data sources.
	They provide the interface that is used by plug-ins to extract data and its structure.
	
	\item \emph{Communication between varied instances of Salomon.}
	Currently the communication is realized using \emph{RMI} technology. 
	The system is homogenic, so using \emph{RMI} seems to be the natural choice.
	In case of appearance of the new implementations it may be necessary 
	to implement communication based on more universal technologies e.g. \emph{CORBA} as well.

	\item \emph{Communication with plug-ins.}
	The group of specialized classes provide data from external data source for plug-ins 
	and enable them performing operations on knowledge. Plug-ins are completely separated from
	the external environment and this layer is the one place when the knowledge exchange 
	between plug-ins and the environment takes place.
	
	\item \emph{User interface.} 
	System \emph{Salomon} may be used in varied contexts, such as interaction with the user, 
	providing services in distributed network environment or as a library used by another program.
	All this forces very strong separation between user interface and the rest of the system.
	
\end{enumerate}

%\begin{enumerate}
%    \item \emph{Komunikacja z zewn�trznymi �r�d�ami danych.}
%    Komunikacja odbywa si� przez specjalne komponenty, kt�re ukrywaj� r�nice pomi�dzy r�nymi �r�d�ami danych.
%    Udost�pniaj� one specjalny interfejs, za pomoc� kt�rego wtyczki mog� pobra� dane oraz struktur� bazy.
%
%    \item \emph{Komunikacja mi�dzy r�nymi instancjami Salomona.}
%    W obecnej wersji systemu do komunikacji wykorzystywana  jest technologia \emph{RMI}.
%    Na obecnym etapie system jest homogeniczny, wi�c naturalnym wydaje by�
%    u�ycie w�a�nie \emph{RMI}. W przypadku pojawianie si� innych implementacji, konieczne mo�e by� zaimplementowanie
%    r�wnie� komunikacji w~oparciu o bardziej uniwersalne technologie jak np. \emph{CORBA}.
%
%    \item \emph{Komunikacja z wtyczkami.}
%    Zestaw specjalnych klas odpowiada za dostarczenie wtyczkom danych z zewn�trznego �r�d�a danych oraz za umo�liwienie operacji na wiedzy. Wtyczki s� ca�kowicie odseparowane od zewn�trznego �rodowiska i~warstwa ta stanowi jedyne miejsce wymiany wiedzy pomi�dzy nimi a~�rodowiskiem.
%
%    \item \emph{Interfejs u�ytkownika.}
%System \emph{Salomon} mo�e by� wykorzystywany w~r�nych kontekstach, takich jak na przyk�ad praca z~u�ytkownikiem,
%udost�pnianie us�ug w~rozproszonym �rodowisku sieciowym czy jako biblioteka wykorzystywana w innym programie.
%To wymusza bardzo mocne oddzielenie interfejsu u�ytkownika od reszty systemu.

%\end{enumerate}

\section{Tests of the system}
%\section{Wykonane testy systemu}

Salomon was tested using freely available data sets from UCI
repository~\cite{uci}. Three plug-ins were use.

%System \emph{Salomon} zosta� przetestowany na powszechnie dost�pnych
%zbiorach danych do testowania algorytm�w uczenia maszynowego
%UCI~\cite{uci}. Przetestowane zosta�y wtyczki zwi�zane z uczeniem
%drzew decyzyjnych.


The aim of the first plug-in was to configure the data processing. It enables
choosing both major and minor characteristics taken into account during building
decision tree. The choice is made with the use of a graphical user interface.
The second plug-in is the actual processing component -� it provides the
realization of the \emph{ID3} algorithm. The last plug-in is responsible for
visualization of the tree created during computation.

%Zadaniem pierwszej z nich jest konfiguracja procesu przetwarzania
%danych. Pozwala ona na wyb�r cechy, wzgl�dem kt�rej tworzone jest
%drzewo decyzyjne oraz cech, kt�re s� brane pod uwag� przy jego
%budowie. Wyb�r ten odbywa si� przy u�yciu graficznego interfejsu
%u�ytkownika. Druga wtyczka jest w�a�ciw� wtyczk� obliczeniow� --
%jest to realizacja algorytmu ID3. Ostatnia wtyczka ma na celu
%wizualizacj� utworzonego drzewa.

These test were quite simple. Currently more complex tasks are
being addressed. One of them is described in the next section.

\subsection{Dalsze prace}
\frame{
	\frametitle{Dalsze prace}
	\begin{itemize}
		\item Mo�liwo�� rozproszenia oblicze�
    	\item Rozdzielanie zada� -- graf przep�ywu
    	\item Wsparcie dla innych typ�w wiedzy
    	\item Wsparcie dla r�nych zewn�trznych �r�de� danych
    \end{itemize}
}

\begin{thebibliography}{99}
\bibitem {bib1} {Michalski, Kaufman, Pietrzykowski, Sniezynski,
    Wojtusiak, Sharma, Seeman, Fischthal, Alkharouf, White, Draminski,
    Glowinski, ''Inductive Databases and Knowledge Scouts''}

\bibitem {bib2} {Kaufman, K. and Michalski, R. S., ''The Development
    of the Inductive Database System VINLEN: A Review of Current
    Research,'' International Intelligent Information Processing and
    Web Mining Conference, Zakopane, Poland, 2003}

\bibitem {bib3} {Michalski, R. S. and Kaufman, K., ''Data Mining and
    Knowledge Discovery: A Review of Issues and a Multistrategy
    Approach,'' Machine Learning and Data Mining: Methods and
    Applications, R. S. Michalski, I. Bratko and M.  Kubat (Eds.), pp.
    71-112, London: John Wiley \& Sons, 1998}

\bibitem {bib4} {R. S. Michalski, ''Knowledge Mining and Inductive
    Databases: An Emerging New Research Direction'', School of
    Computational Sciences, George Mason University, 2004}
    
%\bibitem {weka} Weka -- \href{http://www.cs.waikato.ac.nz/ml/weka}{http://www.cs.waikato.ac.nz/ml/weka}

%\bibitem {yale} YALE -- \href{yale.cs.uni-dortmund.de}{yale.cs.uni-dortmund.de}

\end{thebibliography}

%wylaczenie numerowania na I stronie
\thispagestyle{empty}


\end{document}
